\chapter{Агрессия}

Обезопасить народы от агрессии было в течение последних двадцати лет одним из самых популярных лозунгов демократической дипломатии. Мы установили статус-кво и заявили, что никакие изменения не могут быть внесены, если они не основаны на взаимопонимании всех заинтересованных сторон. Мы <<объявили вне закона>> войну и заявили, что использование вооружённых сил с целью добиться изменения статус-кво без согласия другой стороны является <<агрессией>>. Мы проповедовали, что агрессия является преступлением, и пытались объединить все другие нации (по крайней мере, в рамках Лиги Наций), чтобы они взялись за оружие на стороне жертвы агрессии. Поскольку при данных обстоятельствах не было никаких шансов на какое-либо изменение статус-кво путём соглашений, <<агрессии>> начались в течение нескольких лет после создания этой искусственной мировой конструкции.
 
В течение многих лет, особенно во время заседаний злополучной Конференции по разоружению, широко велись схоластические дебаты с целью получения <<определения агрессии>>. Ни одно из многочисленных предложенных определений не было принято единогласно, но все эти попытки получить определение агрессии имели в основе идею о том, что нация, которая впервые пересекает территориальные границы другой нации с вооружёнными силами, является агрессором.

Такая концепция агрессии в высшей степени неудовлетворительна и создаёт не только полную путаницу в отношении истинного происхождения и начала агрессии, но и парализует свободолюбивые и демократические нации и подвергает их величайшим опасностям.
 
Является ли агрессия преступлением? И если да, то почему?

Агрессия, несомненно, является тем видом человеческой деятельности, который во многих случаях на протяжении нашей истории приносил положительные результаты, подстегивал прогресс и помогал распространению цивилизации и культуры.
 
Прототипом актов агрессии является деятельность миссионеров. Проникновение без приглашения и взаимного согласия на отдалённые континенты с намерением полностью изменить образ жизни незнакомых людей действительно является актом агрессии. Но является ли это преступлением и заслуживает ли оно осуждения?

Крестовые походы были актом агрессии. Французская революция и завоевание Америки Западом были актами агрессии. Но можем ли мы каким-либо образом осудить эти события и назвать их незаконными или преступлением?
 
Представляется очевидным, что концепция, согласно которой начало применения принуждения является единственной формой агрессии и что это во всех случаях агрессия, могла бы быть правильной только в абсолютно совершенном и справедливом мировом порядке, который никогда не существовал и почти наверняка никогда не будет существовать в нашем мире постоянных изменений и эволюции. Неоправданно называть агрессией любой вид насильственного акта, независимо от того, направлен ли он против невинных людей или против несправедливости.

Идея, высказанная одним из лидеров Французской революции: <<Там, где есть порядок и несправедливость, беспорядок является началом справедливости>>, актуальна и в международных отношениях. Таким образом, первое применение вооружённых сил никоим образом не может быть удовлетворительным критерием для термина <<агрессия>>.
 
Если какая-либо нация нападает на одну из наших военно-морских баз или аэродромов, мы называем это <<агрессией>>, и мы готовы вступить в войну, чтобы защитить наши владения. Но когда наши враги нападают на принципы, на которых основана вся наша жизнь, на нашу свободу, наши семьи, наше процветание и наше будущее, мы проявляем безразличие и не доходим до того, чтобы назвать это <<агрессией>> и отреагировать соответствующим образом.

Англия и Франция объявили войну, когда их союзник Польша была захвачена немецкими вооружёнными силами. Соединённые Штаты объявили войну, когда японцы разбомбили американскую собственность на Гавайях и на Филиппинах. Это полностью соответствовало нашей устоявшейся политике, согласно которой война объявлена вне закона, агрессия является преступлением и что агрессия — это первые военные действия, и единственные военные действия.
 
Почти невозможно подсчитать, скольких жизней и скольких достояний и разрушений будет стоить нам эта концепция.

Марш немецких армий и бомбардировки с пикирования японскими военно-воздушными силами были \textit{n}-ным актом агрессии против демократических держав. Называть именно этот \textit{n}-ный акт <<агрессией>> только потому, что он сопровождался фейерверком взрывов пороха является действительно крайне неудовлетворительным.
 
Эта война началась, когда Бенито Муссолини публично и торжественно провозгласил, что целью фашистской Италии является разрушение принципов Французской революции, и когда правительство Адольфа Гитлера публично и торжественно провозгласило, что для нацистской Германии <<Право — это то, что отвечает интересам немецкого народа>>.

Именно в тот момент, если бы наши демократии действовали в соответствии с современными реалиями и если бы у нас было ясное видение реальной собственности, которую мы должны защищать, пробуждённые силы свободных народов должны были вмешаться, чтобы остановить эту агрессию.
 
Защита нашего образа жизни, защита наших наций и уничтожение агрессивных сил обошлись бы в то время менее чем в тысячную долю крови, пота, тяжелого труда и слёз от того, во что это обойдется сейчас.

Перед лицом этой реальности невозможно сказать, что такие вопросы, как: <<Что такое агрессия?>>, <<С чего она начинается?>>, <<Когда мы должны начать борьбу, чтобы остановить её?>> — это всего лишь теория. Фактически, от правильного ответа на такие вопросы и от правильной интерпретации таких теоретических принципов зависят жизнь и смерть миллионов людей, а также сохранение или растрата сотен миллиардов долларов.
 
Вторая большая опасность поверхностного толкования агрессии — именование её просто вооружённым вмешательством — является негативным, оборонительным менталитетом, который она задействует и который она распространяет среди свободолюбивых наций.

Поскольку война объявлена вне закона, военная инициатива является агрессией, а агрессия является преступлением, демократические страны были убаюканы негативной концепцией жизни, которая может только ускорить их разрушение.
 
Мы проповедуем в наших школах, в наших парламентах, в нашей прессе и в наших различных религиозных и политических ассоциациях своего рода пацифизм, который так же далёк от всемирных реалий и от реальной концепции мирного существования, как заклинание от медицинской науки. Мы отказываемся признавать биологические факты, просто не называя их своими именами, и пытаемся вылечить наши болезни благочестивыми проповедями, упрямо отказываясь слушать научно подготовленных медиков и следовать их советам.

Эта странная концепция мирного существования и наша столь же нереальная концепция агрессии создали пассивную, негативную концепцию жизни среди демократических наций, которая включала в себя все проявления нашего существования. Мы ничего не хотели. У нас не было абсолютно никаких намерений, никаких идеалов, никакой цели, кроме как сохранять мир, продолжать отстреливаться и сохранить в неприкосновенности те маленькие земные владения, которые мы унаследовали. Это незнание самых элементарных сил этой планеты сделало нас неспособными не только предотвратить определённые события, но даже справиться с ними, когда они случились.
 
Необычайным зрелищем была та серия сражений, в ходе которых нацистская Германия разгромила одну за другой все европейские нации. Ни одна нация не обладала достаточными умственными способностями, чтобы понять, что происходит, и предотвратить её неуклонный прогресс. Это означало бы инициативу, действие, предотвращение — всё это стало бы невозможным из-за нашей негативной концепции жизни.

Жизнь отдельного человека, как и жизнь нации, — это не что иное, как постоянное действие, инициатива и борьба. Тот, кто остановится, потерпит поражение и должен уступить своё место другим. Невозможно удерживать свою позицию в этом продолжающемся прогрессе и пытаться заставить других остановиться. Концепция статичного мира немыслима, и если мы не поставим динамические силы жизни на службу нашему делу, они будут использованы другими и направлены против нас.
 
Мы сможем не допустить, чтобы эти динамичные силы привели к войнам, только если мы будем готовы направить их в законное русло. Но мы, конечно, не сможем остановить их простым способом — игнорируя их и пытаясь индивидуально убежать от них.

Успехи Гитлера могут быть в определённой степени объяснены правильным анализом последствий этих двух тенденций. В мире, которым почти полностью правят государственные деятели, дипломаты и гражданские служащие, полностью погружённые в статичную, негативную концепцию и оказавшиеся совершенно неспособными действовать и проявлять какую-либо инициативу, внезапно возник необразованный преступник, не имеющий никаких знаний о международных делах, дипломатии, частном или конституционном праве, а просто обладающий простой способностью первобытного человека \textit{действовать}. Не имея контроля над этим примитивным порывом, как это делают наши сверхобразованные, сверхосторожные государственные деятели, он всегда имел на своей стороне силу <<событий>>, которая в тысячу раз сильнее силы аргументов.
 
Если мы хотим извлечь уроки из последних двадцати лет и предотвратить повторение подобных катастроф, мы должны начать с изменения этого бесплодного и пассивного отношения наших народов. Мы должны осознавать, что оборонительные меры на поле боя иногда могут быть стратегической необходимостью; в области идей оно всегда тождественно поражению.

Совершенно неудовлетворительно учить нашу молодёжь и провозглашать нашей общественности, что демократия — это лучшая форма правления, что наше дело — правое и что силы зла будут уничтожены, потому что в конечном итоге правда всегда должна восторжествовать. Такое фаталистическое отношение на самом деле является слабой защитой наших недостатков. Правильно сказать, что правое дело обычно побеждает, но мы можем проследить в истории периоды, когда «правое» дело терпело поражение на протяжении поколений и даже столетий.
 
Более активные и позитивные, а не более праведные, победят в этой борьбе между демократическими и тоталитарными силами. Праведность является фактором победы только в той мере, в какой она даёт нам больше веры и больше сил быть активными и позитивными.

Поэтому наши основные усилия должны заключаться в том, чтобы вырвать из нашей общественной жизни корни этих статичных концепций обороны. Мы должны чётко сформулировать те демократические принципы в национальной и международной жизни, которые мы хотим сохранить, те принципы, которые мы хотим установить и за которые мы готовы бороться. Мы должны дать этим принципам реалистичную и понятную интерпретацию, которая сделала бы возможным их существование и их защиту.
 
Мы должны заявлять об агрессии при любом посягательстве на эти принципы, независимо от того, является ли это нападение военным или идеологическим. И все страны, отстаивающие эти принципы, должны в таких случаях рассматривать себя как объекты агрессии, независимо от того, является ли нападение военным или идеологическим.

Только когда мы сможем чётко определить источник агрессии против нашего образа жизни, и только если мы овладеем механизмом подавления любых подобных атак, мы создадим положение дел, которое можно было бы назвать защищённым от агрессии. В любой международной жизни, организованной на такой основе, военные действия и <<вооружённый конфликт>> будут сведены к минимуму, а когда они всё-таки произойдут, то столкнутся с такой же подавляющей мощью, с какой обычный преступник сталкивается в хорошо организованном городе.
