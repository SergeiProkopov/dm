\chapter{Принуждение}

На протяжении двух столетий германские народы, будь то под руководством Фридриха Великого, принца Бисмарка, кайзера Вильгельма II или Адольфа Гитлера, действовали в соответствии с германской концепцией международного права — <<Сила есть право>>\footnote{%
Здесь и далее переведено дословно, оригинальный текст, использованный автором: \textit{<<Might is right>>}. В обычном употреблении также используются варианты перевода, такие как: <<С позиции силы>>, <<Сила — лучший аргумент>>, <<Сильный всегда прав>>. — Прим. переводчика.}. Эта доктрина была разработана немецкими философами и неизменно практиковалась немецкими государственными деятелями на протяжении последних поколений.
 
Эта доктрина была категорически отвергнута демократическими нациями, которые всегда считали её самым аморальным принципом.

Оппозиция по отношению к теории <<Сила есть право>> увела демократические страны так далеко от неё, что их концепция международной жизни была более или менее проиллюстрирована принципом, который можно было бы сформулировать как <<Право есть сила>> или <<Право без силы>>.

Их отвращение к силе как основе международной жизни было настолько значительным, что они \textit{отвергали} силу и стремились прекратить насилие и принуждение во всех формах.

Трудно сказать, какая доктрина — доктрина <<Сила есть право>> или странная доктрина игнорирования принуждения — более ответственна за нынешнюю ситуацию в мире.

Мы всегда думали, что сила, насильственные действия, принуждение были тождественны войне, а сохранение мира означает только одно: избегание применения принуждения. В истории можно было бы привести десятки примеров, доказывающих, что свободолюбивые нации подвергались нападениям и уничтожались потому, что они не верили в принуждение и считали любое решение предпочтительнее применения силы.

Когда после Первой мировой войны была основана Лига Наций, кардинальной проблемой было должно ли принуждение быть в распоряжении Лиги. Все пацифисты были против этого, и Лига была учреждена без каких-либо положений об использовании принуждения для удовлетворения требований необходимости. Это означало, что с самого начала своего существования Лига функционировала в вакууме нереальности, без каких-либо шансов решить такие вопросы, которые не могли бы быть столь же хорошо решены без её существования.

Правительства с предельным упорством и решимостью отказывались разрешить Лиге применить принуждение. В течение 1931 и 1939 годов было по меньшей мере десять случаев, когда малейшее проявление принуждения, стоящего за резолюциями Лиги, могло бы предотвратить нынешнюю мировую войну. Благодаря самой экстраординарной акробатике рассуждений, каким-то образом это всегда блокировалось. И когда впервые был испытан этот нереалистичный механизм Женевы, когда более пятидесяти стран гневно проголосовали за применение санкций против Италии, ведущие демократические государственные деятели заявили миру, что санкции должны быть безвредными, иначе Европа будет ввергнута в войну.

Было принято достаточно санкций, чтобы вызвать раздражение державы-агрессора, но ни при каких условиях не пропагандировались такие санкции, применение которых повлекло бы за собой применение принуждения. Сэр Сэмюэль Хор сказал: <<Могут быть разные степени агрессии. Эластичность является частью безопасности>>. И когда в целях <<сохранения мира>> демократические правительства высказались за признание итальянского завоевания Эфиопии, лорд Галифакс сказал: <<Какой бы великой ни была Лига Наций, цели, ради которых она существует, превыше её самой, и величайшая из этих целей — мир>>.

Мир был так прекрасно сохранён, и держава-агрессор была так благодарна демократическим странам за то, что они не применили против неё санкций, что менее чем через четыре года Италия объявила войну Великобритании и Франции.

Весь этот печальный опыт должен прояснить нам то, что мы должны были знать без какого-либо опыта: принуждение нельзя отрицать или игнорировать.

\textit{Принуждение есть реальность}.

Если и есть какой-то закон, который можно вывести из истории человечества, так это то, что всякий раз, когда принуждение не применялось на службе закона, оно применялось против закона.

Сегодня мы боремся с чистым бандитизмом. Этот факт больше, чем что-либо другое, парализовал суждения наших демократических государственных деятелей за последние десять лет. Мы привыкли относиться к министрам, послам и представителям других стран как к джентльменам, людям с похожим происхождением и образованием на наших собственных лидеров.

Мы видели гангстеров, организующих ограбления банков, бутлегерство, похищение детей, изготовление и оборот фальшивых денег, но мы никогда раньше не видели и не могли себе представить, что банда гангстеров может завладеть всей государственной машиной и может организовать и управлять великим государством исключительно на гангстерских принципах и методах. Наши демократические политики и дипломаты были беспомощны перед лицом политиков и дипломатов, которые обладали теми же импульсами и мотивами, и которые имели те же представления об обществе, условностях и порядочности, что и обычные гангстеры. И всё же именно это и произошло.

Но почему это произошло? И как это могло случиться? Ответ заключается в том факте, что мы презирали принуждение, мы неправильно понимали принуждение, мы исключали принуждение как инструмент демократической политики. Итак, враги демократии применили принуждение — ничего, кроме принуждения!

Мы не хотели закона с принуждением; просто правила, основанные на доброй воле. Так что теперь нам приходится считаться с принуждением без закона.

До тех пор, пока мы отождествляем принуждение и его применение с войной и верим, что мир — это просто период без применения принуждения, у нас никогда не будет мира и мы всегда будем жертвами принуждения.

Мы не можем отрицать существование дождя только потому, что бывают солнечные дни; и мы не можем отрицать, что есть ночь только потому, что в полдень не темно.

Принуждение — это реальность, которая существует. Мы не можем этого отрицать. Мы не можем называть <<миром>> только те короткие промежутки времени, когда случайно никто не использует принуждения. В любой период, когда насилие не проявляет себя, мы можем, в нынешних условиях, быть уверенными, что принуждение \textit{будет} возрождено.

Только если мы \textit{используем} принуждение, если мы ясно решим, \textit{когда} оно должно быть применено, и если во всех подобных случаях мы увидим, что оно применяется, есть хоть какой-то шанс, что принуждение и насилие не будут применены против наших институтов, нашей свободы и нашей жизни.

Только если мы поставим принуждение на службу правосудию, мы сможем надеяться, что оно не будет использовано против правосудия. Только если мы применим принуждение для поддержания мира, мы можем надеяться, что оно не будет использовано против мира.

Критерием любой организованной формы общественной жизни является то, что вышестоящая власть имеет законное право применять принуждение, чтобы помешать отдельным членам общества применять принуждение.

Единственным способом опровергнуть доктрину <<Сила есть право>> является не <<Право без силы>>, а <<Право, \textit{основанное} на силе>>.

Институты должны быть созданы заранее с той целью, чтобы сила применялась максимально в любое время и при любых обстоятельствах, когда право в опасности. Как сказал Сен-Симон: <<Каждое воссоединение народов, точно так же, как и каждое воссоединение людей, нуждается в общих институтах, нуждается в организации. За пределами этого всё решается исключительно принуждением>>.

При любом планировании будущего, на какой бы основе мы ни хотели организовать отношения между народами, условием успеха является реабилитация принуждения, признание его в качестве обязательной реальности, включение его в нашу будущую схему организации и его независимое использование в заблаговременно оговоренных случаях и в заблаговременно установленных формах.

За то, что христианство пережило двадцать столетий и существует сегодня, мы должны благодарить не только апостолов и святых, но и в равной мере рвение крестоносцев.

Пришло время, когда для выживания демократии нужны больше не апостолы, а крестоносцы, которые убеждены, что мир немыслим ни в какой форме, если он не основан на принуждении.
