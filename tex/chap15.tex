\chapter{Превентивная война}

<<Вы не можете установить мир с помощью войны>>.
 
Это один из самых опасных софизмов, с помощью которых 100-процентные пацифисты внушают свою пропаганду в отношении использования силы, против применения санкций, против любого рода действий, которые могли бы предотвратить развязывание этой мировой войны.

Даже если мы называем применение силы <<войной>>, следует сказать, что \textit{только} такое применение силы может сохранить мир.
 
В чем разница между таким применением силы, которое отстаивается здесь, и войнами, какими мы их знали до сих пор?

Ответ очень прост: наличием закона.

Хотя одной из первых заповедей христианства является <<Не убивай>>, Церковь, когда у неё была власть делать это, приговорила к смерти и казнила тысячи людей, нарушивших законы. И с момента существования национальных государств, даже в самых христианских и демократических из них, применение силы — казнь, тюремное заключение и другие формы наказания — было единственно возможной гарантией христианских или демократических принципов в человеческом обществе.

В этом нет никакого противоречия. И мы должны постоянно помнить о развитии нашей социальной организации, если мы искренне хотим найти решение нашей сегодняшней проблемы: организации международного общества.

Тот факт, что не существует принудительного международного права, создаёт ситуацию, в которой преступники, преисполненные решимости действовать, пользуются всеми свободами и преимуществами беззакония. В нынешней ситуации просто потому, что пресечение или предотвращение преступления не являются результатом применения закона, судебного решения, наказания, они в равной степени представляются нам преступлением.

На самом деле, так оно и есть. Любое наказание без закона приравнивается к преступлению. Только наличие закона создаёт разницу между преступлением и правосудием.

Поэтому первостепенную важность имеет установление принципа \textit{законной войны} как инструмента демократической политики. Только с помощью института законных войн мы когда-либо сможем сохранить мир, и никогда — с помощью альянсов или таких утопических документов, как Устав Лиги Наций или Пакт Келлога\footnote{%
\textit{Келлога — Бриана пакт 1928 (Парижский пакт)} — до­го­вор об от­ка­зе от вой­ны как ин­ст­ру­мен­та по­ли­ти­ки; получил название по именам инициаторов — министра иностранных дел Франции Аристида Бриана и государственного секретаря США Фрэнка Келлога. — Прим. переводчика.}.

Наиболее важной формой законных войн, которую необходимо установить, является \textit{превентивная война}. Точно так же, как в социальной жизни превентивные меры более гуманны и эффективны; точно так же, как тенденция современной медицины направлена на предотвращение болезней, а не только на их лечение, в международных делах мы должны стараться предотвращать крупные вооружённые столкновения, а не ждать, пока начало военных действий станет неизбежным.

До сих пор идея превентивной войны была табуирована в демократических странах. Всякий раз, когда в демократических странах общественное мнение поднималось против поэтапной подготовки тоталитаристами этой войны, фашистские пропагандисты кричали, что демократии \textit{хотят} превентивной войны, поэтому естественным образом они не прекратят действовать так, как они действовали, пока демократии не применят силу. И это было достаточным аргументом, чтобы обезоружить волю к сопротивлению в демократических странах и дать <<умиротворителям>> возможность уладить дела с врагами демократии: капитулировать, а не сражаться.

Никогда ещё война не была более нежелательной и легко предотвратимой, чем та, от которой нам приходится страдать сегодня. Каждому человеку, обладающему здравым смыслом, способному видеть, слышать и понимать, было ясно, что тоталитарные концепции были направлены на уничтожение демократических держав и на мировое господство. Ничто не могло бы быть проще для предотвращения растущей мощи наших врагов, чем проведение превентивной операции, которая до 1938 года была бы незначительной.

Не подлежит обсуждению, что политика демократических держав, заключавшаяся в пассивном ожидании момента, когда эта мировая война станет неизбежной, была ложной политикой. Но бесполезно упрекать демократические правительства за эту губительную политику. Они действовали единственным способом, которым они были уполномочены действовать в соответствии с установленными конституциями, законами и принципами своих стран.

Сможем ли мы когда-нибудь организовать человеческое общество таким образом, чтобы войны стали невозможны и на этой планете воцарился <<вечный мир>>, сказать невозможно. Скорее всего, достичь этого идеала никогда не удастся. Мы должны быть скромнее и реалистичнее, если хотим достичь хоть чего-то, и сделать следующий шаг в направлении этого идеала. И этот следующий шаг — предотвращение мировых войн в масштабах и с разрушительной силой Мировой войны 1914-1918 годов и нынешней.

Это может быть сделано только путём учреждения и узаконивания превентивных войн, как общепринятых демократических действий против роста и распространения антидемократических деструктивных сил, которые должны и всегда будут приводить к незаконным войнам.

Без такого института права никогда не удастся заставить демократические страны осознать серьёзность ситуации, пока не станет слишком поздно. Для простых людей призыв к бездействию всегда намного сильнее, чем призыв к действию. В критические времена в демократических странах всегда найдется подавляющее большинство, которое скажет: <<Это не наше дело>>; <<Давайте не будем вмешиваться>>.; <<Давайте сосредоточимся на защите нашей собственной страны, и к чёрту остальных>>. И тех, кто инстинктивно призывает к действию и предотвращению, всегда будут называть <<разжигателями войны>>, несмотря на тот факт, что действие может предотвратить крупную войну, тогда как бездействие, безусловно, приведёт к войне.

Как только мы учредим такую организацию, у нас, вероятно, будет две или три войны за столетие, но они будут бесконечно малыми по характеру, их опустошение и разрушительная сила будут несравнимо меньше, и соотношение между этими законными войнами к настоящей и прошлой войнам будет таким же, как и сейчас между убийствами среди племен каннибалов и случаями убийств в современном цивилизованном государстве.

Простое введение превентивной войны в качестве законного инструмента международной политики, простая угроза его применения предотвратили бы войны в девяти случаях из десяти и сделали бы использование этого инструмента избыточным.

Единственный вид мира, мыслимый на этой планете и в нынешнем столетии, — это установление конкретных первичных правил между народами и создание вооружённых сил для автоматического и безусловного вмешательства в любую часть мира, где бы эти правила ни нарушались. Такие революционные перемены никогда не могут быть осуществлены без полного признания устаревшего порядка, выявить который могут только катастрофы, подобные нынешней мировой войне. Мы должны надеяться, что эта война донесёт правду о характере и взаимосвязи мира и войны, а также о проблемах, связанных с ними, до сознания масс, и что ясное видение лучшего будущего сформируется во время этой войны, пока продолжаются бойня и страдания. Если мы упустим этот шанс, то поколениям, возможно, придётся ждать другого.

Правосудие началось на этой земле с первой публичной казни преступника по приговору суда.

Мир воцарится на этой земле в тот день, когда впервые группа стран начнет войну, основанную на ранее принятых принципах, против нарушителя международного права.
