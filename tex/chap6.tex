\chapter{Суверенитет}

Золотой телец, которому наиболее преданно и тайно поклоняются массы наших дней — это суверенитет. Ни один символ, претендующий на божественность, который когда-либо овладевал человечеством, не вызывал столько страданий, ненависти, голода и массовых казней, как идея <<cуверенитета нации>>.
 
Что такое суверенитет?

Термин происходит от слова <<суверен>>. В то время, когда народами правили абсолютные монархи, короли, императоры, вожди или как бы они ни назывались, им было необходимо устанавливать свою власть от Бога, чтобы склонить людей поверить и принять, что всё, что они делали, говорили и приказывали, было правильным, непогрешимым и неукротимым. Эти атрибуты назывались <<суверенными>>, а лица, ими наделённые, — <<cуверенами>>.

Многие века люди страдали при такой организации общества, подчиняясь неконтролируемой, верховной власти монархов.

Великий перелом произошёл в восемнадцатом веке, когда под влиянием таких мыслителей и философов, как Локк, Руссо, Монтескье и многих других, массы восстали против своих абсолютных правителей, своих суверенов.

Революционное убеждение состояло в том, что <<суверенитет принадлежит обществу>> и что понятие суверенитета должно перейти от правителя к нации. Эта идея основывалась на опыте древних греков и на концепции Платона, сказавшего, что <<государство, в котором закон выше правителей, а правители ниже закона, имеет спасение>>.

Эта демократическая концепция суверенитета одержала полную победу в девятнадцатом веке и её последствия во всем цивилизованном мире. Трансформация была самой глубокой. Все династии, обладавшие абсолютной властью — Бурбоны, Габсбурги, Гогенцоллерны, Романовы и многие другие более мелкие правящие семьи — были свергнуты; нации приняли республиканскую форму правления; и власть сохранили только те королевские семьи, которые добровольно отказались от своего абсолютного правления и стали символами конституционных монархий, выполняя в государстве функции, аналогичные функциям президентов в республиках.

В то время, когда зародилась идея передачи суверенитета от правителя к нации, промышленная революция ещё не началась, современный транспорт ещё не был изобретен, а термин <<нация>> представлял собой широчайший диапазон, который могли себе представить отцы современных конституций в восемнадцатом веке. Их основная идея состояла в том, чтобы передать суверенитет от \textit{одного} человека \textit{всем} людям — народу, что в то время было тождественно <<нации>>.

По мере того как эта идея обретала форму в современных государствах, она превратилась в нечто совершенно отличное от того, чем подразумевалось. Благодаря развитию техники и коммуникаций, благодаря экономической эволюции географическая территория, на которую распространялся суверенитет наций, становилась всё меньше и меньше. Благодаря политическому развитию многих из этих национальных государств суверенитет — неконтролируемая власть — превратился в институт, который вовсе не обеспечивал народам ту свободу, безопасность и счастье, для которых он предназначался. Напротив, она осуществляла суверенитет способом, не сильно отличающимся от суверенитета монархов. Таким образом, в начале двадцатого века, принимая во внимание весь мир, ситуация была очень похожа на анархическую ситуацию в средние века, когда феодальные землевладельцы осуществляли суверенную власть над своими собственными поместьями, полностью игнорируя интересы общества своих наций, представленные королями.

Суверенитет нации всё больше и больше становился догмой, неизменной, неприкосновенной, неоспоримой, на которой должны были основываться все международные отношения. Все попытки создать какую-либо международную организацию для урегулирования политических, военных или экономических отношений между нациями прискорбно провалились, потому что такая вещь, как мирное сотрудничество между суверенными нациями, немыслима и никогда не будет достигнута.

Мы были свидетелями самых гротескных проявлений этих попыток. Мы проводили различные международные конференции, чтобы снизить тарифные барьеры между странами. Делегаты каждого суверенного государства, естественно, заботились только о наилучших интересах своей собственной страны и старались поддерживать её тарифы как можно более высокими. Если бы они согласились на снижение тарифов в своей стране, они потеряли бы работу. Но сохранение суверенитета своей нации путём отказа от какой-либо уступки означало, что они были компетентными представителями, хорошо заботящимися об интересах своей страны.

В течение многих лет мы следили за дискуссиями на Конференции по разоружению, созванной в связи с настоятельной необходимостью ограничения и сокращения национальных вооружений. Ассамблея состояла из делегатов суверенных государств, каждый из которых подразумевал только защиту своих собственных национальных интересов. У каждого делегата на Конференции по разоружению была одна мысль: сохранить для своей собственной страны максимально возможное количество вооружений. Если бы кто-либо из них согласился на сокращение вооружений своих собственных стран, их сочли бы предателями, действующими в ущерб своим нациям. И когда они вернулись домой, успешно выдержав попытки сократить национальные вооружения, их чествовали как великих патриотов, которые хорошо представляли суверенные права своих стран, чтобы вооружаться без какого-либо <<иностранного вмешательства>>.

Эта комедия вскоре превратилась в трагедию, закончившуюся катастрофой. Несмотря на всё это, суверенитет наций не может быть предметом обсуждения и должен поддерживаться прежде всего и при любых обстоятельствах.

Таким образом, миллионам снова придётся умереть, сотням миллионов снова придётся голодать, и миллиарды долларов снова придётся потратить впустую — потому что мы не хотим признавать, что концепция суверенитета наций, которая была большим прогрессом в восемнадцатом веке, не решила проблему передачи этих суверенных прав от королей народам. До тех пор, пока суверенитет наций имеет только географическое ограничение и пока шестьдесят или восемьдесят наций обладают неконтролируемой властью и суверенным правом набирать армии, объявлять войны, устанавливать тарифы, останавливать миграцию и осуществлять эту суверенную власть над теми правами, от которых зависит всё благосостояние и счастье человечества, то мы не можем сказать, что суверенитет принадлежит обществу.

В тот момент, когда Французская революция материализовала идею суверенитета нации, Франция была величайшей державой в Европе, а её население составляло половину населения всего Европейского континента. В соответствии с условиями восемнадцатого века она была полностью самодостаточным политическим и экономическим образованием. Но в нынешних экономических условиях какой смысл в <<суверенитете Латвии>> или <<суверенитете Люксембурга>>?

Уничтожить золотого тельца суверенитета будет очень тяжелой работой по двум причинам.

Во-первых, корыстные интересы в суверенитете наций огромны. На маленьком европейском континенте в том виде, в каком он был политически организован в 1919 году, в любое время насчитывалось около шестисот членов правительств в звании министра, осуществлявших исполнительную власть. Во много раз больше людей были бывшими министрами и как таковые занимали привилегированные посты в общественной жизни. В любое время по всей Европе было разбросано около семисот-восьмисот действующих послов, полномочных представителей, их превосходительств. Под их началом находилось около десяти тысяч советников, атташе и других чиновников дипломатического ранга. В любое время там было около семи-восьми тысяч законодателей, членов парламентов. Если мы примем во внимание только этих ключевых людей, обязанным своим положением наличию их собственного национального <<суверенитета>>, мы легко сможем понять, что со всеми подчиненными, со всем персоналом, необходимым администрациям суверенных государств, насчитывалось несколько сотен тысяч человек, наиболее мощно организованная каста, которая была полностью процветающими и существующими в соответствии с понятием <<суверенитет>>.

Но, вероятно, существуют ещё более серьёзные корыстные интересы в экономической и финансовой сферах. Одним из катастрофических последствий суверенитета является неправильное представление о самодостаточности или автаркии каждой нации. Были вложены неисчислимые суммы в искусственные отрасли промышленности, созданные и поддерживаемые с помощью тарифных барьеров в каждой стране при полном игнорировании любого экономического закона, исключительно с целью прекращения торговли с другими странами и обеспечения экономической независимости каждой единицы, именуемой <<государством>>.

Вторая причина, по которой будет трудно избавиться от концепции национального суверенитета — и в этом заключается действительная трудность, — носит метафизический характер. Суверенитет нации — это юридическая форма, ритуальное выражение глубочайшего и всемогущего коллективного комплекса неполноценности, который мы называем <<национализмом>>.

Суверенитет и национализм — это две стороны одной и той же фальшивой монеты. Мы не можем принять одну её сторону к нашим услугам, не получив в своё распоряжение также и другую сторону. Обе концепции в их нынешнем виде должны быть уничтожены, и должны быть найдены интерпретации, которые будут ясно выражать в терминах реалий двадцатого века те значения, которые им придавались, когда они были учреждены в восемнадцатом веке.

Помимо ошеломляющей неразберихи и анархии, создаваемых в политической, экономической и международной областях существованием нынешних <<суверенных>> государств, национализм и его правовое выражение — национальный суверенитет, — вызывают наибольшее количество трений, неразберихи и страданий также внутри таких отдельных суверенных государств.

За исключением двух или трех крупных стран, едва ли какая-либо страна в мире состоит из представителей одной национальности, а различные объединения национальностей настолько тщательно перемешаны, что невозможно установить границы таким образом, чтобы все представители одной национальности могли быть расселены в одном суверенном государстве. Следовательно, в каждой суверенной стране, помимо её собственной национальности, существует большое множество других национальностей, так называемых меньшинств, создающих проблемы, которые ни одно суверенное государство никогда не было в состоянии удовлетворительно решить, и являющиеся причиной неизбежных войн.

Если мы оставим в стороне последние эксцессы религиозной нетерпимости нацистских и фашистских государств, мы можем констатировать, что религиозный вопрос был вполне удовлетворительно решен в течение девятнадцатого века в результате отделения религий от суверенитета государства, в результате власти, поставленной выше всех религий, и при которой каждая религия пользовалась равенством.

Но ни одно государство, даже самое либеральное, не смогло решить проблему национальных меньшинств. И эта проблема не могла быть решена, потому что не существовало суверенной власти над национальностями, при которой ко всем им можно было бы относиться как к равным.

Всегда существовала одна из таких национальностей, которая обладала суверенными правами государства по отношению к другим. Итак, мы видели, что в 1914 году Первая мировая война разразилась восстанием тех национальностей и меньшинств, которые чувствовали себя угнетенными в Германской и Австрийской империях. Победа союзников разрушила эти империи и <<освободила>> сербов, чехов, румын и поляков, а также множество других национальностей. Но все эти вновь созданные государства просто представляли собой обратную сторону предыдущего порядка, и было примерно такое же количество людей, которые в 1938 и 1939 годах восстали против чешского, польского, сербского и румынского <<суверенитета>> и тем самым положили начало лавине второй мировой войны.

Та же проблема стоит перед нами в Индии.

Единственное решение и единственная трактовка суверенитета — это предоставить всем национальностям точно так же, как мы предоставили всем религиям полную автономию и полные суверенные права решать свои собственные культурные, национальные и локальные проблемы. Но должна быть создана организация, стоящая выше них, обладающая всеми полномочиями для решения всех этих вопросов — международных отношений, военных и экономических вопросов, — которые должны решаться таким образом, чтобы у каждой нации были равные права и равные обязательства по отношению к ним.

Только посредством такого \textit{отделения суверенитета}, устанавливая национальные суверенитеты по всем национальным вопросам и международные суверенитеты по всем международным вопросам, мы можем создать основу всемирной конституции, которая действительно выражала бы демократическую мысль о том, что <<суверенитет принадлежит обществу>>.
