\chapter{Мир}

Тридцать третья сессия Совета Лиги Наций состоялась в марте 1925 года в скромной обстановке обеденного зала старого отеля <<National>> в Женеве. На повестке дня было принятие Советом Женевского протокола, разработанного предыдущими британским и французским премьерами Макдональдом и Эррио на лужайках Чекерс\footnote{%
Чекерс, Чекерз — с 1921 г. официальная загородная резиденция премьер-министра в графстве Бакингемшир. — Прим. переводчика.}. Надежды на преуспевающую Лигу Наций были всё ещё высоки, и это была первая и единственная серьёзная попытка организовать коллективную безопасность в рамках Лиги.
 
На этом судьбоносном заседании Совета Британию представлял сэр Остин Чемберлен, а Францию — Аристид Бриан. Франция и все представители небольших стран в Совете выступали за принятие Протокола, но все заранее знали, что сэр Остин Чемберлен от имени своего новоизбранного британского консервативного кабинета категорически отвергнет его. Его аргументы заключались в том, что правительство Его Величества не смогло взять на себя такие общие обязательства, которые связали бы им руки в будущем, что целью Лиги было сохранение мира, а не подготовка к войне, и что во всем проекте было слишком много разговоров о возможностях войны\ldots\ <<Правительству Его Величества кажется, — сказал Чемберлен, — что всё, что укрепляет идею о том, что главное дело Лиги — война, а не мир, вероятно, ослабляет её основную задачу — уменьшить причины войны\ldots>>.
 
Бриан, стараясь выглядеть наилучшим образом в сложившейся ситуации, ответил сэру Остину Чемберлену остроумной импровизированной речью, которая создала очень веселую атмосферу для погребения лучшей надежды человечества в период между двумя мировыми войнами.

<<Что такое мир?>> — спросил Бриан. И он дал своё собственное определение. <<Согласно моей философии, — сказал он, — мир — это не что иное, как отсутствие войны>>.

Все мы, присутствовавшие при этом, улыбнулись. Сэра Остина Чемберлена явно позабавили логика и риторика его выдающегося французского коллеги.

Эта дискуссия между двумя величайшими министрами иностранных дел Великобритании и Франции в послевоенный период ясно продемонстрировала, насколько далеки были страны от ясного понимания проблем своего времени.

До сих пор мир действительно был ничем иным, как отсутствием войны, и все усилия дипломатии были сосредоточены лишь на отсрочках и компромиссных решениях любых конфликтов, возникающих между народами. Эта примитивная концепция мира была распространена на протяжении всей нашей истории и особенно в последние годы возвышения национализма и суверенитета.

В течение двадцати лет, предшествовавших этой войне, мы хотели сохранить мир. Мы не хотели ничего, кроме мира. Веря, что мир — это просто отсутствие вооружённого конфликта, и страстно желая сохранить мир, мы были готовы принять любое решение возникших проблем, если только такие решения уберегут нас от вооружённого конфликта.

Таким образом, мы допустили, чтобы подписанные нами договоры рассматривались нашими врагами как клочки бумаги, потому что настаивать на неприкосновенности подписи означало бы развязывание войны. Мы лицемерно закрывали глаза на военную агрессию, потому что думали, что любая попытка предотвратить подобные акты агрессии означала бы вооружённый конфликт. Мы называли интервенцию <<невмешательством>>, потому что думали, что если будем говорить правду, это втянет нас в войну. Мы позволили обмануть себя, подставить под удар и шантажировать, потому что хотели сохранить наш мир. И, наконец, великие демократические державы нарушили даже свои собственные обязательства и клятвы, потому что считали, что мир важнее чести, порядочности и доверия данному слову.

Все ошибки и промахи, совершенные демократическими правительствами за последние два роковых десятилетия, были оправданы доводом о том, что только согласившись с подобными действиями, демократические народы смогут сохранить свой драгоценный мир. У нас не было ни политики, ни идеалов, ни цели, кроме одного — предотвратить стрельбу. Мы не хотели ничего, кроме мира. Так началась война.

Что бы мы ни думали об этой войне, одно несомненно: это бесспорное и убедительное доказательство того, что политика, с помощью которой мы хотели сохранить мир, с треском провалилась.

В том виде, в каком мир понимался до сих пор — период без вооружённого конфликта, — это не что иное, как обратная сторона определения войны Клаузевицем. Более или менее короткие периоды <<мира>>, которыми мы иногда наслаждались в прошлом, были ничем иным, как продолжением войн, ведущихся различными средствами. Все те краткие передышки от войны, которые мы называли <<миром>>, были ничем иным, как дипломатическими, экономическими, политическими и финансовыми войнами между различными группами людей, называемыми <<нациями>>, с той лишь разницей, что эти конфликты, противоборство и враждебные действия велись всеми средствами, кроме настоящей стрельбы.

Если это то, что мы называем миром, если это то, к чему мы стремимся как к идеалу, если это положение дел, которое мы надеемся сохранить <<навечно>> или, по меньшей мере, в течение длительного времени, тогда мир — это утопия, которой мы никогда не достигнем, точно так же, как он никогда и не был достигнут в любой другой период человеческой истории.

Но такая концепция мира крайне примитивна, устарела и нежелательна. В таком статус-кво нет ничего морального, ничего христианского, ничего цивилизованного, ничего демократического, ничего обнадеживающего.

Останется ли мир навсегда утопией или он станет политической реальностью; останется ли он туманной мечтой через тысячу лет, или задачей нашего собственного поколения будет реализовать и организовать его — полностью зависит от того, как мы воспринимаем и как интерпретируем положение дел в этом мире, который мы бы назвали <<мирным>>.

Как чисто негативная концепция, пытающаяся что-то защитить, как просто статичная концепция, пытающаяся сохранить любой территориальный, политический или социальный порядок, существующий или подлежащий созданию, как концепция спокойствия и бездействия, мир невозможен и навсегда останется недосягаемым. На самом деле, если бы такой мир когда-либо мог быть установлен, это прозвучало бы погребальным звоном прогрессу.

Отсутствие войны в течение сколько-нибудь значительного периода между нациями, организованными как суверенные государства, невозможно. Справедливый межнациональный порядок, основанный на суверенитете наций, немыслим, потому что даже если бы мы могли в данный момент установить порядок, который считался бы справедливым всеми заинтересованными нациями, это продолжалось бы недолго, так как суть жизни — это движение и постоянные перемены. Пытаться сохранить мир более чем на короткий промежуток времени между суверенными государствами, обладающими оружием, просто с помощью дипломатии — значит основывать судьбу человечества на искусстве заклинания змей. В борьбе с непреложными реалиями истории нам нужны более эффективные инструменты, нежели флейта.

По мере развития образования будет всё меньше и меньше возможностей удерживать нации от стрельбы друг в друга, и всё больше и больше наций будут претендовать на равные права с другими.

Притязания народов на равенство почти так же стары, как стремление к свободе. Равенство — это ещё один идеал, порождающий очень много путаницы и неправильных представлений.

После своего поражения в 1918 году Германия и другие побежденные нации провозгласили <<равенство прав>> условием мирного развития. В течение десятилетия международная политика, и особенно европейская политика, вращалась вокруг этого принципа <<равенства прав>>.

После долгой борьбы, в декабре 1932 года, этот принцип равенства был дарован странами-победительницами побежденным. С момента предоставления этого статуса равенства ситуация в Европе стала более хаотичной, чем когда-либо прежде, и психоз войны нарастал из месяца в месяц.

Это было неизбежным следствием создания <<основы сотрудничества>>, которая была не чем иным, как фикцией и которая не представляла никакой реальной ценности.

Что обозначает это <<равенство прав>>?

Едва ли в истории есть идеал, которым злоупотребляли чаще и в большей степени, чем идеалом равенства. И за последние десять лет мы своими глазами видели самые чудовищные из этих злоупотреблений. Для Германии требование равенства было не чем иным, как способом империалистического завоевания.

Немцы заявили, что с ними не обращались на равных основаниях, поскольку у них не было права вооружаться, как у французов, поляков и русских, и заявили о праве на равное вооружение. Французы заявили, что ввиду их меньшего населения свобода Германии перевооружаться поставила бы их в неравное положение. Немцы заявили, что Франция обладает богатыми колониями, на что французы ответили, что Германия располагает гораздо более мощным промышленным потенциалом. Немцы повторили, что Франция создала сеть военных альянсов. На что французы ответили, что немцы ведут подготовку всей своей молодёжи для военных целей.

И так далее \textit{до бесконечности} [\textit{ad infinitum}].

И даже когда Третий рейх уже покорил и господствовал над миллионами и миллионами иностранцев, когда у него были самые мощные вооружённые силы, когда-либо созданные какой-то из наций, их лидеры всё ещё требовали <<равенства>>.

Ни у кого не хватило смелости публично заявить, что это были дебаты в вакууме без каких-либо шансов прийти к удовлетворительному решению, потому что это так называемое равенство было не чем иным, как субъективным чувством, интерпретируемым каждым правительством в соответствии с его собственными сиюминутными потребностями.

Равенство — это идеал человеческого разума, а не то, что существует естественным образом. В природе нет такого понятия, как равенство, где более сильный всегда истребляет слабого. Только когда человек осознал, что он создание, превосходящее других животных, родилось равенство как идеал человека.

Но как идеал человека равенство всегда нуждалось в институтах, без которых оно не могло бы существовать. Первыми великими институтами, распространившими идеал равенства, были иудейско-христианские религии с их постулатом о том, что человек был создан по образу Божьему и что все люди равны перед Богом. Этот принцип, провозглашённый много тысяч лет назад, показывает единственную форму, в которой равенство может найти выражение. Это \textit{равенство перед определённой конкретной властью, под конкретным символом}%\footnote{%
%Сравните с Урантийскими Документами: 
%

%\begin{itemize}
% \item 134:4.5 <<У вас не может быть равенства между религиями (религиозной свободы) без религиозных войн до тех пор, пока все религии не согласятся на передачу всего религиозного суверенитета на некий сверхчеловеческий уровень, самому Богу.>>
% \item 134:4.9 <<\ldots\ Концепция равенства никогда не приносит мира, кроме того случая, когда совместно признается некое сверхконтролирующее влияние сверхправительства.>>
%\end{itemize} 
%Прим. переводчика.}
.

На протяжении веков предпринималось множество попыток установить равенство между людьми в политической или социальной сфере. За одним-единственным исключением, они всегда терпели неудачу. Только Французская революция и революции в связи с ней смогли успешно установить законодательство о равенстве, гарантирующее равенство граждан \textit{перед законом}. Такое толкование равенства — равенства перед законом — всегда было самоочевидным в английском общем праве.

Причина этого успеха заключалась в том, что отцы революции придерживались той же практики, что и отцы христианства. Они не хотели устанавливать <<всеобщее равенство>> между людьми, которого не существует и никогда не будет, но они хотели установить равенство в ограниченной и определённой области, в области юрисдикции, и они сделали всех людей равными перед судами, перед законом, точно так же, как христианство сделало людей равными перед символом Бога.

Во многих сферах неравенство в отношениях между людьми осталось неизменным. Всё ещё существовала разница между сильными и слабыми людьми, бедными и богатыми, мудрыми и глупыми, но перед законами государства они были равны.

Что касается общего утверждения о равенстве между нациями, то сейчас мы подошли к той стадии, когда мы должны чётко определить, что мы понимаем под равенством наций. Нынешняя интерпретация <<равенства>>, как права делать то, что делают другие, права вооружаться в той же степени, что и другие, права обладать немного большим количеством оружия, чем есть у других (потому что в противном случае другие могли бы обладать немного большим, и, таким образом, не будет никакого <<равенства>>), является настолько бессмысленной логикой, что нет необходимости тратить время на её обсуждение.

Всегда будет существовать различие между нациями, точно так же, как всегда будет существовать различие между отдельными людьми. И для культурного прогресса крайне важно, чтобы такое различие сохранялось.

Равенство наций — это такой же идеал цивилизации, как и равенство между людьми. Само по себе это противоречит природе и может быть достигнуто только с помощью соответствующей формы устройства.

\textit{Равенство без закона не имеет никакого смысла и вообще никакого морального обоснования.} Это справедливо как для социальной, так и для международной жизни. Только закон делает возможным равенство, и только при наличии чётко определённых законов мы можем сделать нации, как и людей, равными.

В отсутствие закона стремление к равенству в международной жизни является величайшей опасностью и прямой причиной войн.

Без международного права стремление к равенству между нациями ведёт к вооружениям, к союзам, к коалициям. Между двумя соседними нациями одна всегда будет слабее другой. Сначала они соревнуются в увеличении вооружений. Когда эта раса достигает своего оптимума, она начинает искать союзы с другими нациями. Процесс, который повторялся снова и снова в истории, был назван поиском <<баланса сил>>. Это всегда приводило и должно продолжать приводить к войнам.

Равенство без закона означает войну.

Единственный способ поддерживать мир в течение определённого периода без закона — это доминирование одной нации над другой, превосходство одной группы держав над другими.

Единственной возможностью поддержания мира и обеспечения равенства наций является установление закона, согласно которому каждая нация должна быть равной.

Следовательно, кардинальной точкой в определении мира является \textit{закон}.

Только если мы решим принять международные законы, имеющие те же особенности, что и национальные законы, обязательные для исполнения всеми нациями или, по крайней мере, отдельным количеством наций, мы сможем заложить основу для мирного развития международных отношений. Любая концепция о  мире без обязательного международного права является безнадежной мечтой. Критерием любой реалистичной интерпретации мира является её основа на праве.

Мир не является самоцелью. Его нельзя достичь, желая его для себя. Он является наградой за правильную и справедливую политику. Никто из тех, кто не имеет другой идеи, кроме как стать богатым или знаменитым, никогда не будет иметь шанса достичь своей цели. Богатство, слава, комфорт, безопасность достигаются не теми, кто желает их ради них самих; это награда тем, кто активен и полезен, кто производит что-то полезное для других. Они являются лишь сопутствующими результатами таланта, трудолюбия, настойчивости и креативности. Только те, у кого есть творческие идеи и кто концентрирует свои усилия на продуктивной работе, имеют хоть какой-то шанс добиться успеха в жизни.

Если мы хотим мира и ничего, кроме мира, у нас его никогда не будет. Мир — это результат позитивной, созидательной и конструктивной политики. И любая позитивная, конструктивная и созидательная политика сегодня начинается с осознания того, что мы должны отказаться от тех взглядов на фундаментальные принципы международной жизни, которые оказались устаревшими, и дать им толкование в соответствии с реалиями нашего времени.

Обычное отличие национального законодательства от международного права — первое имеет принудительную силу, а второе не имеет такой принудительной силы — является чисто теоретическим определением. Оно не имеет никакой практической ценности.

<<Международное право>>, каким мы его знаем, — это всего лишь система норм, обычаев, правил, договорных обязательств, лишённая принудительной силы. Это вообще не закон. Это игра. Называть это <<законом>> только ещё больше запутывать проблемы международных отношений.

Большинство наших государственных деятелей и экспертов в области права верят, что проблема мира может быть решена на основе такого <<международного права>>. В истории это пытались сделать несколько сотен раз. Пришло время, когда мы должны осознать, что гнались за фата-Морганой.

То, что мы привыкли называть <<международным правом>>, вообще не является правом, и важно избегать использования этого термина при описании нынешних и прошлых международных условий. Мы должны ограничить термин <<право>> мерами, обладающими принудительной силой. И мы сможем говорить о <<международном праве>> только тогда, когда установим систему норм в отношениях между нациями, имеющих такую же исполнительную силу, как и в национальном законодательстве. Учреждение мира на основе того гипотетического международного права, которое мы знали до сих пор и которое является всего лишь обычаем или договорным обязательством, никогда не удавалось и никогда не сможет увенчаться успехом.

Не может быть мирных отношений между нациями без механизма определения случаев <<правонарушений>>, и не может быть мира без варианта ответных мер в отношении таких правонарушений.

Только наличие закона делает определённое действие правонарушением. И только наличие закона делает определённую меру <<санкцией>>.

Признание правонарушения и применение санкций, которые составляют основу любого законного порядка, предполагают наличие закона.

\sloppy История социального прогресса показывает, что применение силы в отношениях между индивидуумами в рамках организованного государства может быть устранено только путём законодательного установления применения силы во всех случаях, когда отдельный член государства совершает незаконное действие.

Международный порядок, основанный на праве, означает именно то, что означает национальный порядок в социальной жизни. Это значит, что применение силы запрещено для индивидуума, но при определённых условиях и в определённых формах оно разрешено для общества.

Введение законного применения силы центральной властью между нациями многими рассматривается как утопия. Однако это единственное решение проблемы. Наоборот, всякая схема, предполагающая решение проблемы мира без законного применения силы в международных делах, является утопической, чему определённо учит история. Это постоянно пробовали. Это никогда не работало. И это никогда не сработает.

Мир — это закон.

Закон является оправданным применением силы — принудительного предписания.

Следовательно, мир без применения силы немыслим.
