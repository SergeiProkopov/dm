\chapter{Взаимозависимость}

Каждый человек рождается в величайшей из всех зависимостей — зависимости от семьи.
 
В начале своей жизни каждый ребёнок совершенно беспомощен, и он полностью зависит от заботы своей матери. Он умер бы в первый же день своего существования, если бы не забота, оказанная ему родителями.

Роль отца в семье — это роль абсолютного монарха.

Каждый ребёнок — раб своей семьи. Он получает от этого свою еду, кров, одежду и должен делать то, что ему говорят родители, не имея права спрашивать почему.

Но как только ребёнок подрастает, его охватывает огромное желание стать независимым от семейных уз. Каждый человек помнит этот период своей жизни как самый неспокойный и непокорный, закончившийся разрывом семейных уз.

Едва человек становится <<независимым>> от семьи, как меняется всё мировоззрение. Независимость не приносит ожидаемого удовлетворения. Вместо этого она вызывает чувство одиночества, сомнения, неуверенности, страха. И через очень короткое время новые побуждения начинают терзать душу молодого человека, как реакция на его независимость: стремление к новым обязательствам, стремление к созданию новой семьи, стремление быть членом общества, профсоюза, партии.

Новое стремление, которое переполняет каждого человека, как только он достигает зрелости, — это стремление стать взаимозависимым со своими соотечественниками.

Эта эволюция — \textit{зависимость, независимость, взаимозависимость} — является естественным жизненным процессом, через который должен пройти каждый индивидуум.

И этот самый процесс представляет собой также этапы, через которые проходят нации в своём историческом развитии. Сейчас мы находимся на том этапе, когда полная независимость принесла наибольшие разочарования большинству наций, когда она рухнула как идеал, и когда у наций возникло своего рода ощущение ограниченности и бесплодности этого идеала.

Переходный период — это ужасный кризис в жизни нации, точно так же, как это трагический момент в жизни индивидуума. Но крайне важно, чтобы мы осознали реальную природу этого кризиса, дабы избежать его использования недобросовестными движениями, стремящимися создать новые формы порабощения наций на том основании, что национальная независимость, как идеал для человека, как она интерпретировалась в течение последних десятилетий, рухнула.

Полная независимость наций в том виде, в каком она была установлена после Первой мировой войны, породила в общности те же чувства, которые полная свобода порождает в жизни человека — чувство сомнения, чувство незащищённости и чувство страха, которые являются причиной вооружения, милитаризма и завоеваний.

Только посредством новой интеграции и взаимозависимой организации наций эти побуждения могут быть преодолены, и национальная свобода и национальная независимость могут найти своё истинное выражение.

На ранних этапах становления нашего общества люди понимали, что свобода может быть практическим институтом для отдельных людей только в том случае, если она ограничена и регулируется законом. Но это свидетельство не было признано в экономической сфере. Здесь, в девятнадцатом веке и первой половине двадцатого века, мы понимали свободу как <<абсолютную свободу>> и долгое время считали мысли и движения, стремящиеся ограничить и организовать экономическую свободу, <<антидемократическими>>. Результатом этого стала растущая экономическая анархия и, несмотря на постоянно возрастающее производство, растущее чувство индивидуальной экономической незащищённости, нищеты и безработицы.

Как реакция на такую абсолютную экономическую свободу, которую мы хотели сохранить, в массах возникло стремление к принуждению, лежащее в основе всех тоталитарных движений. Называть фашистские и нацистские движения <<преступными>> или <<безумными>> — это не объяснение. Они являются естественной реакцией на ложную интерпретацию концепции независимости. Как бы ни было трудно это понять свободолюбивым людям, стремление к принуждению является таким же естественным влечением человеческой природы, как и стремление к свободе, и может быть обуздано только правильной интерпретацией этого идеала.

Полная независимость наций, как мы понимали её до сих пор, не гарантирует свободу наций по той простой причине, что полная независимость наций означает не только то, что нация может делать всё, что она хочет, но и то, что другие нации также полностью свободны делать всё, что им заблагорассудится.

Такая ситуация, далёкая от обеспечения независимости наций, является международной анархией, при которой каждая нация всегда должна быть готова к экономическому разорению или военному вторжению любой другой нации в любое время.

Поэтому очевидно, что гораздо более высокая степень независимости наций была бы достигнута, если бы определённые этапы такой полной независимости были ограничены, регламентированы и должным образом контролировались для всех них, и если бы эти ограничения и предписания применялись ко всем из них институтами, \textit{стоящими над} всеми ними.

Оптимум национальной независимости относителен и основывается на двух факторах:

Во-первых, в какой степени нация сама по себе независима.

Во-вторых, в какой степени нация подвержена вмешательству других независимых наций.

Мы уже видели, что если нация начинает перевооружаться, все другие нации должны сделать то же самое. Если нация переходит к девальвации валюты, все страны в своих экономических отношениях должны сделать то же самое. Если нация заставляет своих работников работать по двенадцать часов в день за мизерную зарплату, никакой социальный прогресс в других странах невозможен. Если нация устанавливает диктатуру, свобода всех других стран оказывается под прямой угрозой.

Перед лицом всех этих фактов последних лет, которые доказывают неоспоримую правильность этого тезиса, цивилизованные люди больше не должны оставаться в стороне, тогда как правительство терроризирует свой народ, преследует миллионы людей, нарушает самые элементарные законы христианской морали, сеет ненависть, перевооружается и готовится к иностранным военным завоеваниям, пока заболевание не станет неизлечимым и не приведёт к общему нарушению.

Должно становиться всё более и более очевидным, что внутренняя система страны определяет её <<внешнюю политику>> и что внутренний режим каждой страны оказывает прямое влияние на все остальные страны.

Наша внешняя политика по-прежнему основана на догме о том, что внутренние дела страны не касаются других стран и что международное поведение правительства — это то, что имеет значение в отношениях между нациями, а не его внутренняя политика.

Это мышление является пережитком тех старых добрых дней дипломатии среди джентльменов, когда разногласия были чисто формальными и внешними. Действительно, конституционные монархии и конституционные республики могут мирно сосуществовать бок о бок и проводить политику невмешательства во внутренние дела друг друга.

Но нации с демократической формой правления и правительства, декларируемой целью которых является повсеместное разрушение демократических принципов; нации, которые уважают договоры и подписи, и нации, которые презирают подобные формальности; нации, которые ненавидят применение силы, и нации, которые поклоняются силе — такие нации не могут мирно жить вместе, если все они являются независимыми и если их независимость никоим образом не контролируется и не ограничивается. В такой международной ситуации война неизбежна.

Разногласия, которые разделяют человечество сегодня и которые вызвали нынешнюю войну, гораздо глубже и фундаментальнее, чем те конфликты, которые в прошлом легко разрешались с помощью поверхностных дипломатических формул.

Ненужные страдания, через которые сегодня приходится проходить человечеству, возможно, позволят организовать международную жизнь на основе взаимозависимости, которая является единственной формой, в которой мы можем найти выход из нынешних потрясений. Такая возможность предоставляется \textit{сейчас}, во время этой всемирной борьбы, в которой участвует каждая нация, и когда каждая нация осознает, что её независимость в том виде, в каком она интерпретировалась до сих пор, не дала им национальной свободы и безопасности.

Мы не решим ни одной проблемы, называя немцев, японцев и итальянцев и их сателлитов гангстерами, гуннами\footnote{%
27 июля 1900 года, во время Ихэтуаньского восстания в Китае, германский кайзер Вильгельм II в своей речи отдал приказ действовать безжалостно по отношению к повстанцам: <<Пощады не давать! Пленных не брать! Подобно тому, как тысячу лет назад гунны под предводительством Аттилы завоевали славу могущественных людей, которая и сейчас живет в легендах, так пусть имя Германии станет известным в Китае так, чтобы никогда снова ни один китаец не осмелился даже косо взглянуть на немца>>. Это сравнение позже активно использовалось антигерманской военной пропагандой союзников во время Первой мировой войны и в меньшей степени во время Второй мировой войны, чтобы изобразить немцев дикими варварами. — Прим. переводчика.} или обезьянами. Проблема в том, как и почему эти гангстеры, гунны и обезьяны смогли стать такими всемогущими, какими они стали на наших глазах. А также как и почему они смогли вызвать пожар на всей планете.

Чтобы заложить фундамент мирового порядка, при котором такие разрушения и катастрофы были бы невозможны, демократические страны должны прийти к <<Декларации взаимозависимости>>, которая должна стать новой Великой хартией вольностей\footnote{%
Великая хартия вольностей (лат. \textit{Magna Carta}) — политико-правовой документ, составленный в июне 1215 года на основе требований английской знати к королю Иоанну Безземельному и защищавший ряд юридических прав и привилегий свободного населения средневековой Англии; является договором между королем и баронами, по которому король обещал не брать с баронов никаких платежей без согласия <<общего совета королевства>>, в состав которого входили бароны; среди прочего, Хартия провозглашает, что ни один свободный человек в Англии не может быть арестован, заключён в тюрьму, лишён имущества и изгнан из страны иначе как <<по законному приговору равных или по закону страны>> [by the legal judgement of his peers or by the law of the land], устанавливает принцип неприкосновенности личности; бароны получали законное право начать военные действия против короля, если он не будет соблюдать Хартию; в силу этого данные статьи получили название конституционных.  — Прим. переводчика.} человечества.

В этой Декларации взаимозависимости мы должны ясно изложить конкретные элементарные принципы социальной, политической и экономической жизни, которые каждая нация должна принять. Мы должны ясно заявить, что нарушение любого из этих правил и принципов повлечёт за собой немедленные действия со стороны коллективного органа других стран или любой другой коллективной международной организации.

Такое изменение в международной жизни, регулирование национальной независимости, было очевидно предвидено сторонами, подписавшими Декларацию независимости в 1776 году. В ней говорится: <<Что всякий раз, когда какая-либо форма правления становится разрушительной для этих целей (жизни, свободы и стремления к счастью), народ имеет право изменить или упразднить её и учредить новое правительство, основанное на таких принципах и организующее свои полномочия в такой форме, как им должно представляться, что это вероятнее всего окажет воздействие на их безопасность и счастье>>.

Этот самый небезопасный и несчастный период в истории, эта катастрофа, в которой мы оказались, настоятельно требуют, чтобы мы прислушались к советам Отцов Независимости и устранили проблему в корне.
