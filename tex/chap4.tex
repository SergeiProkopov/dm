\chapter{Нация}

С национал-социализмом к власти пришла новая концепция, которая находится в абсолютном противоречии с идеями, на которых до сих пор основывалась международная жизнь всех государств, демократических или автократических. Очень серьёзное изучение этой новой концепции необходимо для того, чтобы избежать окончательного разрушения демократического миропорядка и для того, чтобы иметь возможность проводить реалистичную политику и перестать заниматься политикой в вакууме.
 
Этой разрушительной концепцией, которая в течение нескольких лет оказала такое разрушительное воздействие, является национал-социалистическая теория о <<нации>>. Очень важно точно знать, что мы имеем в виду, когда говорим о <<нации>>, поскольку это отправная точка всех политических действий в отношениях между народами.

Согласно демократическому тезису, принятому всеми западными странами и неоспоримому в течение прошлого столетия, нация — это совокупность населения, состоящего из представителей любых рас, верующих в любые религии, говорящих на любых языках, объединенных одним и тем же идеалом в одном и том же государстве.

\sloppy Национал-социализм заменил эту теорию другой, согласно которой <<нация>> (\textit{Volk}) — это совокупность людей одной расы, одного языка, одного происхождения \textit{<<Volkstum>>} — независимо от государства или государств, в которых они проживают.

Эти две концепции в основе противоположны друг другу.

Демократическая концепция чётко определяет ячейки международной жизни и позволяет разрабатывать политику, основа которой является фиксированной. Этими ячейками являются структуры — государства. Согласно этой теории, нации: <<Великобритания>>, <<Германия>>, <<Соединенные Штаты Америки>>, <<Франция>>, <<Бразилия>> чётко определены. Все международные акты, конвенции, договоры, союзы, организация отношений между разными народами могут основываться на хорошо обоснованных компонентах.

\sloppy Нацистская теория полностью переворачивает этот установленный порядок. <<Рейх>> — это, конечно, такое же государство, как и прежде, и подобное другим, но оно не идентично юридической форме <<немецкой нации>>. Естественно, пруссак или баварец по-прежнему является немцем, но согласно этой теории румыны в Трансильвании, американцы в Миннесоте и бразильцы в Сан-Паулу, если они немецкого происхождения, в равной степени принадлежат к <<немецкой нации>>, то должны хранить верность немецкому рейху и должны находиться на службе у немецкого правительства.

Согласно этой теории, нация — это физическая и психологическая конструкция и, по сути, расовая и культурная общность, а не политико-территориальная конструкция. Таким образом, немецкая нация не обусловлена немецким государством, поскольку <<нация>> и <<государство>> — это два совершенно независимых понятия.

Гитлер говорит в \textit{Mein Kampf}, что немецкий рейх как государство охватывает всех немцев не только для того, чтобы сохранить расовые элементы этой нации повсюду, но и для того, чтобы поднять немецкую нацию до доминирующего положения во всем мире. Министр внутренних дел Германии Фрик в брошюре, предназначенной для немцев, проживающих за границей, говорит, что каждый американец немецкой крови должен всегда помнить, где бы он ни оказался, что он является частью Германии. А нацистский теоретик Альфред Розенберг в своей книге \textit{Myth of the Twentieth Century} заявляет: <<\ldots бесчестная демократия, безрасовая концепция государства девятнадцатого века должна быть полностью уничтожена\ldots>>
\footnote{%
Здесь цитата исправлена и приближена к оригинальному тексту \textit{<<Der Mythus des 20. Jahrhunderts>>}. Для сравнения приводится оригинальный текст цитаты, использованный автором: \textit{<<\ldots``the infamous conception of the state of the nineteenth century must be crushed.''>>}. — Прим. переводчика.}.

Естественно, ученые, государственные деятели и общественное мнение в демократических странах отвергли эту национал-социалистическую доктрину о нации, но они отвергли её с милостивой улыбкой и с той тщеславной самоуверенностью, которая так характерна для нашего поколения. Демократические нации не хотели принимать всерьёз эти <<необоснованные>> взгляды и не верили, что они каким-либо образом угрожают их институтам и их существованию. И когда эти теории были применены на практике в Германии, они думали, что это внутреннее дело Германии, которое никоим образом не может повлиять на внутреннюю жизнь и политику демократических народов. Они также твёрдо верили, что Германия и другие тоталитарные державы с их концепциями <<нации>> вполне могли бы существовать и жить бок о бок с демократическими странами с их демократически-правовой концепцией <<нации>>.

На самом деле это было то же самое, что позволить волкам и овцам жить в одном загоне. Принятие нацистской концепции нации позволило нацистскому правительству создать свободную от ограничений организацию граждан немецкой расы в Дании и Бразилии, в Югославии и Чили, во Франции и Соединенных Штатах, а также практически во всех странах мира. Эта организация подразумевала не просто поддержание культурных связей с Отечеством. Она подразумевала воспитание этих иностранных граждан немецкого происхождения в национал-социалистическом и антидемократическом смысле; подготовку их в военных формированиях; обучение их шпионской и диверсионной работе, привитие им идей и воспитание враждебных настроений по отношению к государствам, в которых они проживали; превращение их в военные ударные отряды против стран, в которых они были, в демократическом смысле, национальными гражданами.

Этот подрыв демократических наций стал настолько сильным, что стал одним из главных орудий тоталитарных держав в их мировой захватнической войне против демократий, и во многих случаях этого было достаточно, чтобы подготовить страны-жертвы к военной оккупации без какого-либо значимого сопротивления. Демократические державы были совершенно беспомощны и неспособны защитить себя от этой разрушительно мощной военно-политической машины, которую национал-социалистическое правительство, основанное на расовой концепции нации, выстраивало в течение нескольких лет совершенно открыто и заметно.

Препятствовать американскому или французскому гражданину говорить и писать то, что он думает противоречило конституционной гарантии свободы слова, и в соответствии с существующими демократическими принципами, было невозможно провести различие между лояльными демократическими гражданами и такими гражданами, которые явно были врагами государства и агентами антидемократических сил. Демократические правительства были в состоянии проводить различие только между <<гражданами>> и <<иностранцами>>, поскольку они были определены в соответствии с существующими демократическими законами, какими бы неадекватными и неудовлетворительными они ни были. В демократических странах ничего не предпринималось в этом отношении, пока не стало слишком поздно, и тогда в моменты отчаяния единственное, что делалось, — это облавы на <<иностранцев>>, большинство из которых были самыми страстными борцами за демократию и первыми жертвами нацистских политических концепций.

Ничего нельзя было сделать в отношении тех многих тысяч датских, американских, югославских или французских <<граждан>>, которые изо всех сил трудились и боролись за разрушение страны, в которой юридически они были гражданами, а также за победу нацистской Германии, на чьей службе — сознательно или несознательно — они находились. Если небольшое число этих <<граждан>> было арестовано, то это было связано с тем, что так или иначе можно было возбудить дела, основанные на предъявленных им обвинениях, которые отличались от их реальных преступлений.

Напрашивается вывод, что такое положение дел, при котором демократическое государство может быть подорвано и разрушено изнутри его собственными гражданами исключительно посредством нынешней интерпретации демократической концепции нации, является неприемлемым и внутренне противоречивым. Мы должны дать чёткое и недвусмысленное определение понятиям <<нация>> и <<национальный>>, если мы хотим предотвратить полное разрушение демократических институтов повсюду.

Демократическая концепция нации является единственным политическим следствием христианских принципов. Согласно христианским принципам, каждый человек создан по образу Божьему и равен в глазах Бога, если он принимает символы христианства и авторитет Церкви. Следовательно, следует рассматривать как аксиому организованной демократии то, что все граждане должны быть равны при условии, что они соблюдают законы своей страны и признают авторитет демократической конституции. Это возмутительное убеждение, что мужчины должны быть низведены до уровня скотоводства и классифицироваться в соответствии с цветом волос, количеством зубов или происхождением их бабушек.

Но демократическая концепция нации, которая наделяет каждого гражданина равенством и правом пользоваться всеми привилегиями этого статуса, может быть только для тех индивидуумов, которые со своей стороны принимают эту демократическую концепцию и авторитет законов, выражающих эту концепцию. В демократическом государстве нет места ни одному индивидууму, который не на сто процентов лоялен этим демократическим концепциям. Отречение от них, проповедь расовых теорий или любых других взглядов, противоречащих демократической концепции нации, пребывание каким-либо образом на службе у иностранных правительств, отстаивание нацистско-фашистских принципов должно рассматриваться как преступление самого серьёзного характера, наказуемое в мирное время так же строго, как и государственная измена во время войны.

Демократическая концепция нации должна быть чётко интерпретирована, и такая интерпретация должна быть представлена в законодательстве, которое обеспечивало бы свободный и демократический образ жизни всем тем, кто верит в свободный и демократический образ жизни, но которое налагало бы суровые наказания на любого, кто использует свободу и демократию в качестве своих инструментов в борьбе за разрушение свободного и демократического образа жизни.

Без такой новой интерпретации и законодательства демократические нации находятся полностью во власти своих врагов — фашистских тоталитарных государств, — у которых в нынешних условиях есть все возможности для разрушения демократии изнутри. Ограничение иммиграции или контроль над иностранцами — очень наивная попытка предотвратить эту опасность. Все руководящие сторонники пятой колонны и предатели, все те, кто действительно обладал властью помогать нацистским и японским завоеваниям, были национальными гражданами своих собственных стран. Немецкие, итальянские или испанские политические беженцы были в последний момент отправлены в концентрационные лагеря, но Лаваль, Бодуэн, де Бринон, Квислинг, Мюссерт, Тисо, а также тысячи их пособников считались патриотами до тех пор, пока они не нанесли последний удар по свободным институтам их собственных народов.

Крайне важно, чтобы те немногие страны, которые всё ещё наслаждаются демократией, полностью осознали реалии кризиса и поняли, что Лаваль, Квислинг и Антонеску находятся у власти не потому, что они были преступниками и предателями, а потому, что действующие законы их соответствующих стран, пока они были демократиями, не предоставили законных средств демократическим правительствам для предотвращения их подрывных антидемократических действий и поступков, а также дали им полную свободу уничтожать свободу.
