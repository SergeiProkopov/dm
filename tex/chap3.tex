\chapter{Либерализм}

Политической мыслью, которая пыталась воплотить в реальность идеалы свободы в социальной жизни, был либерализм. Либеральные идеи были ядром программ, которые привлекли наиболее влиятельные силы в конце восемнадцатого века одновременно в Соединенных Штатах, в Англии и во Франции. Конституция Соединенных Штатов, Французская революция и начало современного индустриализма в Англии придали огромный импульс этим идеям, и прогрессивные элементы всех демократических стран были объединены в политические партии, целью которых было организовать государственную и экономическую жизнь на основе индивидуальной свободы, защищать независимость наций и создавать как можно более широкую свободу для международного обмена.
 
Теперь либеральные партии повсеместно находятся в упадке, их влияние исчезает, а власть ослабевает.
 
Поражение либеральных партий — это в высшей степени экстраординарное явление. На это есть две причины. Первая, довольно парадоксальная, — это принятие либерально-демократических принципов почти всеми остальными политическими партиями в демократических странах. В Англии и Консервативная партия, и Лейбористская партия стали <<либеральными>>. Таким развитием событий они лишили Либеральную партию главных аргументов. Такие же постепенные изменения можно наблюдать и во всех других демократических странах. И Демократическая, и Республиканская партии в Соединенных Штатах являются <<либеральными>>.

Второй причиной упадка либеральных партий является не менее парадоксальный факт, что они автоматически становились антилиберальными, превращаясь во всё более и более доктринерские партии с конкретными и жёсткими программами, которые они защищали с наибольшим алиберальным фанатизмом. Эта реакционная тенденция либеральных и радикальных буржуазных сил находится в полном противоречии с первоначальными идеями либерализма, нацеленными на политическое и социальное освобождение индивидуума для того, чтобы открыть каждому человеку и каждой нации свободный путь к эволюции.

Если мы изучим политические взгляды и партии демократических стран, независимо от того, есть ли в стране только две или три партии, или двадцать или тридцать различных партий, мы можем разделить все эти явно расходящиеся программы и партии на две группы, в которые мы можем включить все существующие и возможные политические взгляды.

В первую группу мы можем отнести все те партии, движения и группы, которые верят в осязаемую, чётко проработанную программу. С этой точки зрения революционные и реакционные партии принадлежат к одной и той же группе. Между ними нет никакой разницы ни с политической, ни с теоретической точки зрения, за исключением фактора времени. Программа наших нынешних консервативных партий была революционной двести лет назад, а программа наших нынешних революционеров, если они её осуществят, будет реакционной сто лет спустя. Те принципы и программы, которые сегодня горячо отстаиваются Республиканской партией в Соединенных Штатах, консерваторами в Англии и правыми буржуазными кругами во Франции, считались революционными до Американской и Французской революций, а их сторонники считались опасными элементами в государстве, многие из них казнены как таковые.
 
С другой стороны, Коммунистическая партия и коммунистическая программа, которые мы считаем наиболее революционными силами нашего времени, становятся консервативными, как только они создаются. Каким бы ни было наше мнение о коммунизме, режим Большевистской партии в России является прототипом консервативного режима. Они самым фанатичным образом придерживаются конкретной и устоявшейся системы, защищая её всеми средствами. Они не терпят никаких противоположных взглядов, и любого, кто открыто пытается что-либо изменить в существующем устоявшемся порядке, ждет та же участь, что постигла либералов при самодержавии, и участь, с которой сталкиваются сами коммунисты во многих демократически-капиталистических странах.

Ко второй категории мы можем отнести все такие партии и политические движения, которые не верят в совершенство какой-либо закрытой системы и обстоятельной программы, будь то уже существующие или планируемые к созданию, но которые верят в постоянный прогресс, в необходимость постоянных изменений в социальных, политических и экономических институтах для того, чтобы адаптировать их к реалиям в каждый данный момент времени. Этого убеждения придерживались различные либеральные и радикальные партии с момента их зарождения.
 
История подтверждает, что эта вторая политическая концепция ближе к реальности, поскольку мы не можем проследить в прошлом ни одну политическую или социальную систему, которая была реализована в её чистом и целостном виде, и мы не можем обнаружить ни одного периода, в течение которого не происходили бы изменения.



Согласно принципам либерализма,
должна существовать свобода слова, свобода собраний, свобода прессы, свобода голосования, право избирать представителей. Эти свободы были предоставлены всем, и знаменитые слова Вольтера: <<Я полностью не одобряю то, что вы говорите, и буду до смерти защищать ваше право говорить это>>, — были самоочевидны.

Такое положение дел, предоставляющее каждому человеку полную свободу слова, свободу печати, свободу собраний и предоставляющее каждой ассоциации и партии полную свободу деятельности и представительства в парламенте, является логическим результатом триумфа идеалов Французской и Американской революций и британских реформ.
 
К сожалению, в этой концепции не учитывался только один элемент — элемент Человека. Её теории и институты были основаны на абстракции <<homo democraticus>> точно так же, как либеральная школа экономической мысли была основана на абстракции <<homo economicus>>. Эти абстракции довольно хорошо работали в теории и на практике до тех пор, пока широкие массы нового либерально-демократического порядка ощущали только преимущества новой эры по сранению с \textit{дореволюционным режимом} [\textit{ancien régime}], который всё ещё был жив в их воспоминаниях.

Но как только экономическая и политическая эволюция породила один кризис за другим, одну крупную войну за другой; как только стало ясно, что либерально-демократические системы в том виде, в каком они были созданы, не могут решить проблему распределения и могут лишь частично решить проблему производства; как только стало очевидно, что повышение уровня жизни после установления либерально-демократического режима не было постоянным; как только политические и националистические силы вмешались в свободный товарообмен и миграцию, целые классы и нации не увидели дальнейших шансов на улучшение почвы существующей системы. Как только массы осознали эти проблемы, элементы <<homo economicus>> и <<homo democraticus>> отреагировали не так, как предполагали их создатели.
 
Догматический либерализм, который пронизывал все основные политические партии, все правительства, всю государственную службу и весь дипломатический корпус демократических держав, оказался совершенно неспособен справиться с новой ситуацией. Недовольные элементы начали публично нападать на демократический порядок. Против них ничего не было сделано, потому что существующий демократический порядок гарантировал каждому гражданину <<свободу слова>>.

Всё больше и больше людей, недовольных существующим порядком, собирались вместе и начинали организовывать и проводить митинги против устоявшихся принципов демократии. Никаких шагов по их пресечению предпринято не было, поскольку действующие законы гарантировали <<свободу собраний>>.

Озлобленные и разочарованные интеллектуалы, солдаты, банкиры и промышленники, которым никогда не нравилась форма народного правления, присоединились к оппозиционным группам, поддерживали их финансово и социально, и вскоре мы увидели книги, брошюры, журналы и газеты, издаваемые и распространяемые с единственной целью борьбы с существующим демократическим порядком. Против них не было принято никаких мер, поскольку они работали на строго законной основе, которая гарантировала им <<свободу прессы>>.

Движение становилось всё более влиятельным, принимая форму политических партий, которые добились избрания в парламенты. В каждом законодательном собрании представители появлялись во всё большем количестве с единственной целью — бороться в парламентах против института парламентаризма. Против них ничего не было предпринято, потому что их действия были гарантированы конституционными правами.

Таким образом, либерально-демократические государства были подорваны, их общественная жизнь отравлена, а их институты в конечном итоге уничтожены полностью <<законными>> методами, сначала в Италии, затем в Германии, далее в одной стране за другой, распространяясь подобно лесному пожару.

Возник жаркий спор между истинно верующими в демократические принципы и теми правящими кругами, которые постоянно отступали и уступали одну демократическую позицию за другой перед безжалостно наступающим врагом. Аргумент всегда был один и тот же: <<Мы являемся демократией; наши граждане имеют гарантированные конституцией права, и мы не можем действовать против них антидемократическим образом>>.

Поэтому они ретировались. И когда они оказались перед выбором — капитулировать или изменить свои методы, их догматизм укоренился настолько глубоко, что они предпочли капитулировать.

\sloppy Пагубные последствия либерально-демократического догматизма очевидны для всех. Трагедия в том, что либеральные лидеры были настолько ослеплены своим теоретическим догматизмом, что отказывались понимать современные проблемы даже после того, как проиграли большинство крупных сражений, и защищали свои абстрактные концепции из последних сил с убеждённостью, достойной лучшего применения.

Бесчисленны публичные высказывания побежденных либеральных лидеров, пытающихся оправдать свои позиции в этой битве. Их неизменные взгляды лучше всего были выражены в одной фразе, написанной лидером английских либералов лордом Сэмюэлем в эссе, опубликованном в \textit{The Nineteenth Century and After}, ещё в июне 1938 года. Лорд Сэмюэл выступил против тех, кто призывал к изменению методов в наших отношениях с диктаторскими державами, заявив: <<Быть либеральным только по отношению к либеральным государствам — это вовсе не либерализм>>.

Одна эта фраза выражает нереальность догматического либерализма, который в основном ответственен за парализацию наших правительств, наших дипломатов и наших государственных служащих.

Если мы попытаемся проанализировать эти демократические принципы, то наиболее простой логикой окажется то, что в организованном обществе свобода слова не может означать свободу слова для тех, кто хочет отменить свободу слова.

Свобода печати не может означать свободу печати для таких изданий, которые стремятся подавить свободу печати.

Свобода голосования не может означать свободу голосования для тех, кто хочет отменить свободу голосования.

Места в парламентах были предусмотрены не для тех, кто хочет уничтожить парламентаризм.

Демократические институты были созданы не для тех, кто хочет упразднить демократию, и они не могут быть в распоряжении этих людей.

Эти простые выводы кажутся самоочевидными и почти аксиоматичными при обсуждении принципов демократии.

Но если существуют определённые люди, которые не хотят принимать результаты логического мышления, им следует читать и перечитывать историю всех демократических стран за последние двадцать лет. Им придётся осознать, что демократия была уничтожена повсеместно средствами, предоставленными их завоевателям самими демократиями. В этом мире реальности любая система, каждая нация должны погибнуть, если у них ничего нет, кроме благочестивых проповедей ненасилия, с помощью которых они могли бы противостоять суровым силам.

Победа тоталитарных диктатур и поражение демократических институтов вовсе не означают поражения демократических принципов, но они означают полное поражение той конкрентой интерпретации демократических принципов, на которой мы основывали своё существование в течение последних десятилетий.

Мы думали, что достаточно провозгласить определённые правила среди либерально и демократически настроенных людей, чтобы жить в условиях демократии — что-то вроде эксклюзивного клуба, где мы можем жить среди джентльменов, соблюдая правила и устав клуба.

Но наш эксклюзивный социальный порядок был низвергнут огромным количеством новоприбывших, единственной целью вступления в клуб которых было разграбить кухню и жульничать за покерными столами.

Если мы хотим установить демократический порядок не в вакууме и не на основе абстракции <<homo democraticus>>, то мы не можем исходить из предположения, что самая важная составная часть в нашей конструкции — элемент, Человек — будет реагировать каким-либо иным образом, чем он должен в соответствии со своими качествами и природными условиями.

Демократия, как и любая высокая форма культуры и цивилизации, является в определённой степени роскошью, которую могут понять и оценить только люди с определённым образованием и которая может быть установлена только после удовлетворения некоторых более примитивных потребностей. Поскольку немыслимо, чтобы в наше время все люди обладали пониманием, знаниями, видением, характером и моральной способностью действовать и реагировать так, как должны действовать и взаимодействовать демократические люди, мы можем заставить демократию работать, только если будем готовы создать необходимую защиту для демократического режима, предписывая и провозглашая в недвусмысленной форме то, что мы понимаем под каждой из свобод, предоставляемых режимом, и если мы создадим одновременно с предоставлением этих свобод необходимые институты для их защиты и предотвращения их разрушения.

Основной принцип правильной интерпретации либерально-демократических институтов заключается в том, что демократические институты создаются для всех тех, кто придерживается демократических взглядов, и единственно возможная интерпретация либеральной политики заключается в том, что к каждому человеку, каждой партии и каждой нации следует обращаться в соответствии с их собственными принципами.

Это единственно возможная интерпретация, которая позволила бы нам реорганизовать демократический мировой порядок таким образом, чтобы демократические народы, преданные демократическим институтам, были свободны от нападений и разрушений внутри своих собственных стран или со стороны внешних сил.

Свобода слова означает, что каждый человек имеет право выражать свои мысли и иметь полную свободу говорить всё, что он хочет сказать, за одним-единственным исключением: он не должен иметь права говорить, что он против свободы слова.

Свобода печати означает, что каждый человек имеет право печатать книги, периодические издания или газеты и публиковать в таких печатных средствах массовой информации любую идею, которую он пожелает, за одним исключением: он не должен иметь права печатать или обнародовать то, что свобода печати — это плохой институт, который должен быть упразднен.

Все люди должны иметь свободу собраний, и на таких собраниях каждый человек должен иметь свободу выражать любые взгляды, которых он может придерживаться, за одним-единственным исключением: он не должен иметь права нападать или критиковать право на свободу собраний.

Каждая политическая партия должна иметь право быть допущенной в законодательные собрания, и каждый должным образом избранный представитель должен иметь право свободно выражать свои взгляды по общественным вопросам, за одним-единственным исключением: ни один спикер не должен иметь права нападать на институт парламентаризма и представительного правительства, и ни одна партия, отстаивающая такие взгляды, не должна быть допущена к участию в парламентских дебатах.

Конечная цель демократии — это, естественно, демократический человек, но ввиду того факта, что человеческая раса, даже самые высокоцивилизованные нации, не полностью состоят из демократических людей, мы должны обеспечить демократию для тех, кто настроен демократически, и нашей первичной демократической обязанностью является противодействие антидемократическим мыслям, движениям и силам, использующим антидемократические методы.

Мы давным-давно поняли, что единственный способ гарантировать безопасность жизни мирных граждан — это создать организацию, которая отделит тех людей, которые вмешиваются в мирную жизнь других. И мы также хорошо знаем, что единственный способ гарантировать сохранность личной собственности граждан — это создать организацию для лишения личной собственности тех лиц, которые наносят вред личной собственности других.

Невозможно построить работоспособную демократию ни на национальном, ни на международном уровне без защиты политических прав демократического народа такими принудительными институтами, которые предотвращали бы и ограничивали деятельность тех элементов, которые по какой-либо причине действуют против демократических привилегий демократических граждан или наций.

Только если мы достаточно чётко определим, что мы понимаем под каждой из демократических свобод, которые наши государства могут предоставить индивидуумам, и только если мы создадим необходимый организм для предотвращения и подавления всех таких сил, которые работают против этих демократических привилегий, и только если мы примем доктрины, что либеральная политика подразумевает быть либеральным по отношению к тем, кто либерально мыслит, и относиться ко всем индивидуумам, партиям или нациям в соответствии с их собственными принципами, можем ли мы надеяться, что в будущем мы будем способны в любой момент остановить такие движения, которые представлены в нашем поколении нацизмом и фашизмом? Только тогда мы сможем надеяться, что у нас будет демократический порядок, который будет работать в соответствии с нынешними реалиями нашего мира.
