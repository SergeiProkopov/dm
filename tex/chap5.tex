\chapter{Национализм}

Великий кризис, который опустошает мир с начала двадцатого века, который разразился в больших масштабах в 1914 году и третью фазу которого мы сейчас переживаем, ещё недостаточно чётко объяснен. Различными мыслителями и писателями было предложено великое множество экономических, политических, финансовых, технических и моральных объяснений. Все они правы со своей конкретной точки зрения, но все они опираются лишь на неполный феномен и не дают представления о реальном происхождении этой мировой катастрофы, которую, несмотря на сложность её последствий, обнаружить не очень трудно.
 
За последние 150 лет два мощных потока захлестнули человеческую расу и с непреодолимой силой потащили её в разные стороны.

Одним из таких направлений является промышленная эволюция. Колоссальное развитие промышленности, начавшееся в начале девятнадцатого века и призванное поднять материальное благосостояние человечества до уровня, о котором до тех пор и не мечтали, имеет своим основополагающим и господствующим характером высокомерную тенденцию к \textit{универсализму}. Ускоренный ритм экономики, распространение её активности по всему миру, массовое производство, рационализация, коммуникации и обмен не являются изобретением какого-то либерала или интернационалиста. Всё это является необходимым условием и основой любого увеличения богатства.

Этот исторический процесс свободно протекал своим естественным ходом до того момента, когда население Европы выросло до таких размеров, что для народов стало невозможным существовать по старым экономическим формулам, не снижая уровень жизни. Завоевания, достигнутые в девятнадцатом веке, которые были результатом промышленного прогресса, несомненно, были решающими и представляются уникальным явлением в мировой истории. Столетие научной работы и технического прогресса плюс международная организация экономики просто превратились в насмешку над старой теорией, которая предсказывала истощение природных ресурсов и объявляла голод в ближайшем будущем неизбежной участью постоянно растущей человеческой расы.

Обеспечение человечества товарами, которые используются в настоящее время, и предоставление достаточного количества свободного времени, с пользой используемого для обучения, были до мировых войн технически и экономически гарантированы, несмотря на постоянный рост населения до такой степени, что это было бы невозможно представить за сто лет до этого.

Поддержание и продолжение этого прогресса на пути к материальному благополучию, экономической свободе и культурным достижениям зависит от разделения труда, строго соблюдаемого не только в центре каждого предприятия, но и в различных отраслях промышленности и среди наций, что означает распределение производства в наиболее подходящих местах при соблюдении свободы обмена и свободы миграции.

Примерно в то же время, когда начался этот уникальный процесс промышленной эволюции, на основе концепции Французской революции был задуман новый идеал, который всегда глубже укоренялся в человеческой душе и разуме до такой степени, что стал самой могущественной религией нашего времени — национализм. По самой своей природе национализм, как он понимается и культивируется сегодня, стремится к \textit{обособлению}, \textit{дифференциации} и всё более жёстко делит человечество на малые единицы.

Эти два мощных течения — интегрирующая эволюция индустриализма и дифференцирующая эволюция национализма — доминирующие в нашу эпоху, действуя как огонь и вода, сейчас бурно сталкиваются, а взрыв и сильный пожар, через которые мы проходим, являются последствиями этого потрясения.

Кризис, с которым мы боремся — это кризис национализма и индустриализма. Он разразился в июле 1914 года и завершится либо крахом западной индустриальной цивилизации, либо уничтожением национализма как основы политики. Такова альтернатива, которая нам предлагается.

Национализм — результат Французской революции — изначально был высоким идеалом человечества. Его целью было освобождение народов от доминирования абсолютизма, провозглашение их независимости, передача символа суверенитета от короля народу и достижение социального порядка, основанного на принципах равенства, свободы и справедливости.

В конце восемнадцатого века национализм в том виде, в каком он был задуман первыми основателями современной демократии, был огромным шагом вперед. Это означало расширение фундаментальных принципов государства с одного человека или малой группы на всю нацию. Это было основой индивидуальной свободы, верховенства закона, свободных выборов, представительного правительства.

Но однажды утвердившись в качестве основного принципа политики, национализм постигла та же участь, что и все другие замкнутые революционные идеалы, как только они перестали быть идеалом и стали реальностью. <<Суверенитет нации>> — огромное достижение 150-летней давности, когда промышленный прогресс был ещё на ранней стадии развития, — начал наносить ущерб реалиям экономической жизни во второй половине девятнадцатого века. И с тех пор, подобно всем социальным идеалам, которые становятся догмами, это стало величайшим препятствием на пути дальнейшего прогресса. Это стало известной участью некультурных масс, выражением самых низменных инстинктов массового комплекса неполноценности, а его защитники — самыми нетерпимыми служителями культа догматической религии, которой мы когда-либо обладали на этой земле.

Национализм — это не политическая концепция. Он больше не представляет собой человеческий идеал. Он является главным выражением влиятельных интересов. Он обладает всеми критериями строго догматической религии, глубоко укоренившейся в душе — глубже, чем все дисциплины, которые мы охотно называем религиями. Идеалы и символы национализма, такие как понятия <<родина>>, <<флаг>>, <<национальный гимн>>, являются типичными табу, к которым сегодня в высокоцивилизованных странах прикасаться опаснее, чем к табу диких каннибалов Южных морей. Ни один человек, ни одна партия не осмелится прикоснуться к этим святыням; никто не смеет их критиковать. Тем не менее, следует сказать, что их возвышенный культ является одним из главных корней зла нашего времени.

Если человек говорит громко и публично пять раз в день: <<Я величайший человек в мире>>, все будут смеяться над ним и подумают, что он сумасшедший. Но если он выражает тот же самый психопатологический импульс во множественном числе и публично говорит пять раз в день: <<Мы величайшая нация в мире>>, то он обязательно будет считаться большим патриотом и государственным деятелем, и он вызовет восхищение не только своей собственной нации, но и всего человеческого рода.

В течение прошлого столетия три различные организации безуспешно пытались бороться с национализмом: католическая церковь, либеральные движения и международная организация рабочих.

Вечный идеал христианства (который выражается в вере в единое божество и в постулате мира на этой земле) находится в абсолютном противоречии с идеалом национализма — этой современной религии, которая в действительности основывается на разделении человечества на несколько групп в соответствии с происхождением, расой, языком и суверенитетом, в соответствии с благоговением каждого народа к определённому богу.

Положение дел, преобладающее в настоящий момент в мире, с религиозной точки зрения было бы справедливо охарактеризовано термином <<политеизм>>. Современная религия национализма вытеснила христианскую веру из души человека, и хотя он ходит на мессу и посещает церемонии христианской церкви, настоящим богом, которому он предан превыше всего, в которого он верит, за которого он готов сражаться и отдать свою жизнь, является не единый Бог всеобщего христианства, а богиня — <<нация>>.

Эти боги очень напоминают языческих богов дохристианской эпохи. Они настаивают на признании своей расы и ненависти к другим расам. Они настойчиво требуют войны и победы, и требуют отмщения, если вместо победы приходит поражение. У Англии есть свой бог, как у Франции, как у Соединенных Штатов, как у Германии и Италии; у русских и чехов есть свой бог, как и у поляков, аргентинцев и японцев. Безразлично, насколько мала нация, у неё есть свой собственный национальный бог.

\sloppy Все эти нации скрывают свои языческие инстинкты под маской христианства. Во всех странах национализм рассматривается как <<христианская политика>>. Повсюду языческий дух культивируется под моральной защитой ложно истолкованного христианства. На всех полях сражений, где бушуют войны, христианские священники маршируют перед войсками, неся символ Сына Божьего — того Сына Божьего, который искал мира и любви, — и одной и той же формулой они благословляют два противоборствующих лагеря, готовых обрушить ярость своей национально-языческой страсти.

Бесспорно, что такое положение дел должно быть неприемлемым для каждого настоящего христианина, и также бесспорно, что Церковь больше всего заинтересована в том, чтобы её великий универсальный идеал — монотеизм — был реализован не только на небесах, но и на земле среди народов.

К сожалению, христианские церкви, напуганные прогрессом науки, промышленности и либерализма, думали, что их материальные интересы совпадают с теми силами, которые противостоят человеческому прогрессу. Напуганные эксцессами индустриальной и демократической эволюции, характеризуемой доктринами масонства и религиозными преследованиями в Советской России, они всё больше и больше примыкали к националистическим силам, пока не обнаружили, что в этой великой человеческой борьбе они объединились во многих частях мира с силами фашизма, сама суть которого является антихристианской. Хотя политика Церкви по сути консервативна и антиреволюционна, лидерам христианской веры вскоре придётся осознать, что они собираются разрушить сами принципы христианства, если по чисто материальным соображениям будут отождествлять себя с различными национализмами, которые сегодня борются во имя своих особых национальных богов. Ни при каких условиях земным следствием идеалов человека, созданного по образу Божьему, милости, терпимости и милосердия не могут быть расовые преследования, зоологический материализм, концентрационные лагеря и культ агрессивного милитаризма.

Второй силой, которая пыталась предотвратить катастрофические последствия догматического национализма, были либеральные элементы, которые в конце девятнадцатого и начале двадцатого веков были довольно влиятельны среди буржуазии всех стран. Эти групы понимали, что положение дел, порожденное национализмом, не может быть прочным, и они стремились преодолеть национальные антагонизмы и объединить народы посредством соглашений, договоренностей и взаимопонимания. Эти тенденции получили сильное развитие в годы, последовавшие за Первой мировой войной, и их великим достижением стало создание Лиги Наций.

В течение определённого периода, когда Бриан и Штреземан всецело поглотили международную арену, можно было почти поверить, что они добьются своих целей. Но они потерпели неудачу, справедливо пораженные железным законом, потому что существуют антагонизмы, которые при самой доброй воле в мире и даже при самой искусной дипломатии не позволят себя сломить, и потому что невозможно примирить в договорах и соглашениях в сущности непримиримые силы.

Некоторые народы, прежде всего англо-американские демократии, даже сегодня придерживаются этого идеала, который они считают осуществимым: добровольное и мирное сотрудничество различных суверенных государств, опирающееся на <<добрую волю>> народов. Они упрямо отказывались предпринимать какие-либо усилия по созданию более единой международной организации, что привело бы к ослаблению национального суверенитета, международному законодательству, международным силам, обязательствам, гарантиям и санкциям. Они провозгласили единственной основой совместной работы <<добрую волю>> народов, существующих или порождаемых.

Этот идеал полон противоречий. Если бы так называемая добрая воля народов существовала или была возможна в качестве основы международных отношений, то действительно было бы бесполезно и ненужно изменять или улучшать нынешнюю организацию этого мира. Никакое соглашение, договор или закон не были бы нужны, если бы действия людей или наций основывались на том, что мы представляем себе под термином <<добрая воля>>.

Но независимо от такого упрощения рассуждения, ни у кого нет причин сомневаться в том, что каждая нация движима всей возможной доброй волей и что она не имеет в виду ничего такого, что она не считала бы праведным, когда заседает с другими нациями за столом переговоров. Остается фактом, что, несмотря на это, механизм не работает.

Причина достаточно ясна. Идея <<доброй воли>> — это плод воображения, из которого ничего нельзя извлечь. Гёте в своём \textit{<<Фаусте>>} определяет природу дьявола как <<часть той силы, что вечно хочет зла и вечно совершает благо>>. Человек находится на прямо противоположном полюсе по отношению к Мефистофелю. Он является частью той силы, что вечно хочет блага и, тем не менее, вечно совершает зло.

Нет людей — если бы хоть один существовал, он считался бы редким исключением, — действия которых продиктованы исключительно злыми побуждениями; так скажем, которые не желают добра себе, но желают зла своему ближнему; которые не желают защищать своё собственное дело, которое они считают праведным, но которые просто хотят разрушить дело других. Величайший преступник совершает преступление не для того, чтобы причинить вред другим, а для того, чтобы обеспечить себе преимущество и осуществить побуждение, которое в данный момент кажется ему правильным. Несмотря на эту <<добрую волю>> людей, которая, несомненно, существует, социальный порядок невозможно представить без законов универсальной силы и без вынужденного подчинения индивидуумов этим законам.

Нет никакой разницы между сообществом людей и сообществом наций. Ни одна нация, взятая в отдельности, не желает причинять вред другой. Все служат своим <<обоснованным национальным интересам>>; все <<защищают свою страну>>; все хотят <<защитить себя от агрессии>>. Именно эта глубокая убеждённость в правоте своих собственных интересов и потребностей привела к фактическому хаосу и сделала невозможным его разрешение. Мы должны попытаться ясно разглядеть эту сложную взаимосвязь побуждений и поступков. Мы должны действовать в рамках политики реальности, признавать существующие факты и справедливо соизмерять возможности.

<<Сближение позиций>> между различными суверенными государствами, основанное на принципе национальности и движимое национализмом, невозможно. Это чистая утопия. Все попытки международного сближения сталкивались с влиянием национализма, который не допускает никаких важных уступок, ни политических, ни экономических, без которых <<международное сотрудничество, основанное на доброй воле>>, не может быть реализовано.

Человек должен выбрать: либо он придерживается концепции национального суверенного государства, которая неизбежно ведёт к изоляции, автаркии и, в конечном счёте, к конфликтам и войне; либо он желает создать международную или, по крайней мере, континентальную или региональную организацию, которая могла бы гарантировать мир и способствовать экономическому прогрессу. В последнем случае нужно отойти от религии национализма и её международных последствий.

Либеральные силы, которые в последние десятилетия становятся всё более догматичными, вряд ли будут в состоянии осуществить это. Они потеряли свою власть в области внутренней политики, потому что превратили идеалы либерализма в негибкую и догматичную консервативную программу, тем самым дав своим противникам оружие, с помощью которого можно их уничтожить. Их взгляды на организацию международной жизни настолько же противоречат их собственным принципам, насколько они предоставляют тем нациям, которые противостоят их взглядам, возможность и средства уничтожения демократических народов, которые всё ещё придерживаются фундаментальных идеалов либерализма.

Третьей и, возможно, самой важной силой, которая боролась с национализмом, был <<Интернационал>> социалистических рабочих партий. Пролетариат давным-давно осознал, что его освобождение может быть достигнуто только путём объединения, и он организовал своё движение на международной основе. Следуя девизу <<Пролетарии всех стран, соединяйтесь!>>, должна была быть создана влиятельная всемирная партия. Она должна была осуществить во всех странах социалистическую программу в соответствии с директивами коммуны.

Уже перед Первой мировой войной социалистические партии были высокоразвиты и добивались успехов. Они добились значительного прогресса после войны, когда почти во всех европейских странах распадались буржуазные партии с милитаристской, консервативной доктриной. Миллионы людей, не только рабочие, но и представители крестьянства, среднего класса и интеллигенции, принадлежали к социалистическим партиям, только от которых они ожидали панацеи. После многих лет социалистического развития и господства эти партии пришли в упадок почти во всех странах. Повсюду недовольные массы заставляли их отказываться от превосходной позиции, завоевание которой заняло у них так много времени.

Политика социалистических партий за последние двадцать лет привела к величайшему обману широких масс. Великое народное движение, которое вначале породило все надежды, было разрушено, когда оно столкнулось с суровостью реальности. Причина, по которой социалистические партии не смогли даже частично реализовать свои программы и почему их последователи так быстро отказались от них, заключается в том, что <<Интернационал>> был всего лишь фикцией.

Во всех странах, где социалистические партии приходили к власти, они придерживались националистической политики. Все попытки, предпринятые с целью реализации экономической политики социалистической программы, провалились, как только они стали зависимы от национализма, и если что-то и было достигнуто, то это были лишь немногие достижения социальной политики, которые Бисмарк или Ллойд Джордж могли бы с такой же легкостью внедрить, как и социалистические лидеры.

С другой стороны, они предполагали <<международное сближение позиций>>, задуманное в духе либеральных мыслителей, выдающих желаемое за действительное. Они никогда не осмеливались бороться с национализмом в своей собственной стране, или не хотели этого делать. Последствием этой политики, направленной на преодоление антагонизмов, которые не могли быть преодолены, стало отсутствие успеха у социалистических правительств и тех, кто поддерживает социализм. Ультранационалистический и милитаристский дух пережил через несколько лет поразительное возрождение. Экономический кризис принял катастрофический характер. Рабочие, истинным представителем которых должна была быть партия, оказались в ещё большем бедственном положении. Безработица быстро росла, и не только сторонники, но и многие из первоначальных лидеров социалистических партий перешли в лагерь национал-фашистских партий.

Психология этого процесса крайне проста. Власть, осуществляемая на протяжении долгих лет, не увенчалась успехом. Лидеры, которые были националистами — но не настолько, чтобы конкурировать с настоящими посланниками национализма, — свалили всю вину на <<иностранцев>>, на другие нации, которые вмешивались в их политику. Вполне естественно, что массы быстро последовали за демагогами, которые со всей силой своей ненависти боролись с <<другими нациями>> и своим собственным демократическим правительством, слабым и не имеющим никакого успеха в своём активе. Социалистические партии потерпели неудачу, потому что они не понимали, что на современном этапе промышленного развития все проблемы, породившие социальный вопрос, могут быть решены только на международном уровне.

Итак, в последние годы мы были свидетелями триумфа национализма над христианством, либерализмом и социализмом — над всеми теми силами, которые противостояли ему.

Сегодня миром повсеместно правят националистические силы. Разделение между нациями и их изоляция были доведены до абсурдного предела, и вряд ли возможно идти дальше в этом направлении. Это исступление национализма означает его конец. Характерно, что даже один из величайших представителей современного национализма, Бенито Муссолини, когда он ещё был способен выражать свои собственные взгляды, чувствовал, что национализм на нынешней стадии политического и экономического развития больше не является стимулятором. Перед нападением на Эфиопию он написал в статье следующие фразы: <<Национальность как принцип и как связующее звено в уравнивающем формировании стала великой динамичной силой, которая способствовала росту современной Европы. Однако эта сила вскоре перестанет быть центростремительной; она угрожает превратиться в центробежную силу и стать фактором разъединения, если её не поддерживать в состоянии разумного равновесия. Невозможно проявить полную человеческую справедливость по отношению к каждому человеческому инстинкту и гордости каждой расы. Соединенные Штаты показали, что старые чувства ненависти можно трансформировать в национальное образование за одно или два поколения, если расы, которые раньше были движимы атавистической враждой, передаваемой от отца к сыну, объединятся>>\footnote{%
Отсутствует источник подтверждающий достоверность данной цитаты. — Прим. переводчика.}.

Национализм фактически подошел к началу своего конца. Он разрушил и испортил всё, что создал человеческий разум, всё, что было создано человеческим трудом. Абсурдность национализма лучше всего характеризуется тем фактом, что сегодня мы обладаем техническими средствами, позволяющими пересечь Атлантический океан за семь часов, но уходит семь месяцев на получение визы.

Национализм стремится не к объединению человечества на основе 95 процентов его общих характеристик, а к разделению его на основе 5 процентов его различий. Это духовная концепция, находящаяся в абсолютном противоречии со всеми победами, достигнутыми в течение последнего столетия. Это убеждение, которое хочет возродить дохристианскую эпоху. Это враг человека и антихристианство. Он реакционен и делает невозможным какой-нибудь прогресс на пути к благоденствию. Он порожден террором и страхом, подозрительностью, недоверием и тщеславием. Это самая злая эпидемия, которая когда-либо нападала на человечество. Это выражение коллективного комплекса неполноценности.

Это не имеет абсолютно ничего общего с благородным чувством, с любовью к собственной стране, с настоящим патриотизмом, искажением которого оно является, точно так же, как патологическое пьянство является извращением удовольствия от бокала вина.

Необходимо понимать, что существуют только две реальности — \textit{индивидуум} и \textit{человечество}. Все остальные классификации на касты, племена, классы, религии, расы и нации являются произвольными, искусственными и поверхностными.

Никто не может сказать, что такие разделения человечества по какому бы то ни было признаку являются фундаментальными с какой бы то ни было точки зрения. Французы, немцы и итальянцы — швейцарские патриоты. А африканские негры, желтокожие китайцы, краснокожие индейцы, голубоглазые ирландцы, темноволосые греки и светловолосые скандинавы гармонично живут вместе, испытывая одинаковое чувство преданности своей стране — Соединенным Штатам Америки.

Естественно, мы не можем сделать кожу негра белой и не можем отказаться от использования немецкого языка. Но мы можем упразднить принцип, согласно которому именно эта необоснованная классификация человечества на расы или национальности остается основой суверенных государств. Как только мы поймем эту проблему и усмирим принцип национальности как основы государств, националистические войны прекратятся так же автоматически, как религиозные войны прекратились в тот момент, когда религия была отделена от государства и перестала быть его основой.
