\chapter{Свобода}

Великим двигателем человеческой истории является борьба за свободу. Практически все войны велись за свободу. Все революции начинались ради свободы. Все человеческие начинания в научной, экономической и технической областях проистекают из стремления к большей свободе. Идея свободы лежит в основе практически всех идеалов в политической и социальной сферах, за которые люди борются сегодня, точно так же, как они боролись на протяжении тысячелетий. И всё же ни одно представление не создало большей путаницы, чем идея свободы.
 
Что такое свобода на самом деле, мы до сих пор не знаем, и она широко открыта для интерпретации. Эти различные интерпретации понятия <<свобода>> являются причиной той неразрешимой путаницы, в которой различные нации, различные идеологии, различные партии и различные классы ожесточенно противостоят друг другу. Каким бы трудным это ни казалось, мы должны постараться как можно яснее определить, что такое свобода, если мы хотим использовать это понятие так, чтобы мы могли приблизиться к нему, а не делать его непригодным.
 
К сожалению, мы привыкли при обсуждении политических, социальных и религиозных принципов пользоваться диалектическим методом. У нас пока нет лучшего, более абстрактного и более научного метода исследования и обсуждения.
 
Этот диалектический метод, который мы обнаруживаем уже очень развитым в трудах греческих философов, оперирует противоположностями, антитезисами. Мы говорим, что свобода — это антитезис принуждению, что мир — это антитезис войне, что независимость — это антитезис обязательствам и так далее. Чтобы понять, насколько опасно идти по этому пути, мы должны помнить, что этот диалектический метод философии, после его наивысшего выражения в диалогах Платона, привел прямиком к софистической школе.

Если мы рассмотрим все эти представления, на которых основано каждое противоречие по политическим и социальным проблемам, мы легко увидим, что то, что мы считаем противоположными концепциями, на самом деле находится на одном уровне и в разной степени выражает идентичные проявления.
 
Свобода без равенства — непостижимое положение дел. Поскольку равенство между людьми, нациями или между любыми другими человеческими объединениями явно противоречит природе — оно никогда не существовало и, вероятно, никогда не будет существовать — свобода в её чистом и полном понимании привела бы к положению дел, которое было бы точной противоположностью любому виду свободы.
 
Если бы мы дали каждому человеку — сильному и слабому — и каждой нации — большой и малой — полную свободу действий, не налагая никаких ограничений на их побуждения, это привело бы к величайшему террору, угнетению, насилию — к полной анархии.
 
Таким образом, очевидно, что тот вид свободы, который мы рассматриваем как человеческий идеал, является своего рода синтезом между свободой и принуждением. Тот факт, что какая-то внешняя сила запрещает мне убивать человека, который мне не нравится, или отбирать собственность у тех, у кого её больше, чем у меня, значительно ограничивает мою свободу. Но то же самое ограничение защищает меня от убийства теми, кто меня не любит, и от ограбления теми, кто завидует тому, чем я владею. У меня определённо есть ощущение, что защита от убийств и краж \textit{усиливает} мое чувство свободы в большей степени, чем то, насколько то же самое ограничение \textit{лишает} меня свободы, запрещая мне совершать те же действия в отношении других.

Эту очень простую и очевидную взаимосвязь можно было бы проиллюстрировать любым количеством примеров. Но независимо от того, сколько примеров мы можем привести, представляется очевидным, что свобода и принуждение имеют функциональную взаимосвязь и не могут рассматриваться как противоположности. Таким образом, идеал свободы, как мы его понимаем, является совершенно относительным представлением, зависящим от двух факторов: во-первых, от того, в какой степени человек может действовать свободно; и, во-вторых, в какой степени он подвергается свободным действиям других. Только при правильном синтезе этих двух факторов может быть достигнут наилучший вариант и мы сможем добиться такого положения дел среди людей, которое можно было бы назвать свободой.

Эта тесная взаимосвязь между свободой и принуждением в социальной жизни людей была признана в самом начале нашей цивилизации. Самые примитивные формы социальной жизни начинались с запрета определённых действий. Основанием христианской религии являются Десять заповедей, чётко устанавливающих десять действий, которые <<Ты должен\ldots>> или <<Ты не должен\ldots>>.\footnote{%
Здесь переведено дословно, для сравнения приводится оригинальный текст автора: \textit{<<\ldots\ ``Thou shalt\ldots'' \ or ``Thou shalt not\ldots''>>}. В английских вариантах перевода Библии используются именно эти фразы в тексте Десяти заповедей, в русских же переводах такие дословно переведённые фразы отсутствуют. — Прим. переводчика.}

Таким образом, мы можем утверждать, как бы парадоксально это ни звучало, что свобода в истории человечества началась с первого законного введения принуждения.
 
Эти первоначальные и вековые принуждения были ограничены теми самыми примитивными импульсами людей, запрет которых, очевидно, подразумевает больше свободы для общества, чем их свободное осуществление.

От Десяти заповедей до самого запутанного законодательства наших дней существует одна четкая линия: только посредством юридических ограничений свободного проявления человеческих импульсов мы можем достичь состояния дел, которое мы можем назвать <<свободой>>.
 
В социальной жизни это самоочевидно, и никто, кроме анархиста, не стал бы притворяться, что законы, запрещающие убийство, кражу, лжесвидетельство, подделку чеков или перерасход денег с банковского счёта, являются законами, противоречащими принципу свободы. Мы все знаем, что с прогрессом цивилизации наша социальная жизнь становится всё более и более сложной, и что эта эволюция требует всё большего и большего принуждения по отношению к человеческим действиям, и что только совокупность этих принуждений может дать нам свободу.

Крайне важно чётко понимать взаимозависимость между свободой и принуждением, если мы хотим решить все те многочисленные политические, экономические и международные проблемы, с которыми мы сталкиваемся.

Особенность заключается в том, что эта взаимозависимость между свободой и принуждением, которая является вековым опытом нашей социальной жизни и которая на протяжении всей нашей истории была базовым принципом великих законодателей и основателей религии — дабы привнести некоторую систему в социальные взаимоотношения между людьми и создать максимально возможную индивидуальную свободу — до сих пор не принята и не признана в качестве базового принципа в политической сфере и в международных отношениях между нациями.

В этих двух важных сферах, в которых бушует нынешний кризис, мы по-прежнему утверждаем, что свобода и принуждение противоречат друг другу, что всякое принуждение противоречит принципу свободы и что установленные свободы могут поддерживаться только без какого-либо принуждения. Отсюда и анархическая ситуация, в которой мы живем.

Демократические страны провозгласили великим историческим достижением свободу слова, свободу печати, свободу собраний и многие другие политические свободы, которыми в равной степени должны пользоваться все граждане. Эти свободы были предоставлены без каких-либо ограничений и без какого-либо чёткого определения, без каких-либо запретов, в результате чего в большинстве стран свобода слова, свобода прессы, свобода собраний и все другие свободы были уничтожены только и исключительно потому, что эти свободы были предоставлены в абсолютной форме, в уверенности, что любой запрет или принуждение будет противоречить принципам предоставленных свобод.
 
Тот факт, что во многих странах в нашем поколении эти свободы, столь дорого оплаченные веками борьбы за них, были уничтожены только и всецело свободой действий этих демократических привилегий, предоставленных врагам свободы, является достаточным доказательством того, что провозглашённые в своей абсолютной форме эти свободы не гарантируют свободы.

Нет никакого смысла в том, что свободная и демократическая страна должна предоставлять каждому неограниченную свободу и все демократические средства для борьбы со свободой и демократией как таковыми. Демократы-доктринёры считают, что такое условие присуще принципам демократии, и предоставление свобод в разной степени разным людям противоречит этим так называемым, но так и не обозначенным демократическим принципам. Это очень простая и самоочевидная точка зрения, если мы используем представление о том, что свобода и принуждение противоположны. Но такой образ мышления в корне ошибочен. В общественной жизни нации взаимоотношение между свободой и принуждением точно такое же, как и в социальной жизни, и будем ли мы пользоваться политической свободой или нет, будет полностью зависеть от правильного толкования этих принципов.
 
Проблемы, касающиеся международных взаимоотношений народов, носят точно такой же характер. Основываясь на абсолютной концепции свободы, мы пришли к концепциям национальной независимости и национального суверенитета как высшим проявлениям свободы государства в его взаимоотношениях с другими государствами. Мы считаем, что любое ограничение этих абсолютных концепций суверенитета и независимости противоречило бы идеалу свободы нации. Именно благодаря этой концепции абсолютной свободы на международной арене создалась такая же анархическая ситуация, какую абсолютная свобода создала бы в социальной жизни любого сообщества — так много наций подверглось нападениям, потерпело поражение и завоевано, при использовании грубой силы в качестве единственного арбитра между нациями, и сотни миллионов людей снова стали рабами.
