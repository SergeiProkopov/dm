\chapter{Утопия}

Эти страницы не содержат ни одной новой идеи. Вряд ли возможно выдвинуть какую-либо действительно новую и оригинальную идею относительно политического и социального устройства мира. И, конечно, нет необходимости в какой-либо новой идее для решения наших нынешних международных проблем.
 
Принципы надлежащего управления были открыты столетия назад. Основные идеи, касающиеся человеческого общества, разработанные Конфуцием, Платоном, Аристотелем и, более конкретно, философами восемнадцатого века, можно рассматривать как самоочевидные. Единственная проблема заключается в их правильной интерпретации, их постоянной эволюции, их согласовании с современными реалиями и их применении через такие институты, которые могут придать им наибольшую силу и наилучшее выражение.

Все крупные конфликты, все революции и все войны происходили из-за того, что существующие институты, преобладающая интерпретация принципов и методов политической процедуры вступали в противоречие с постоянно меняющимися реалиями. Полагать в любой момент, что важна уже существующая форма, а не содержание, должно и всегда будет приводить к катастрофам.

Трагедия человечества в том, что, как сказал Гёте, ни одно поколение не живет по законам своего времени. Нам всегда приходится нести бремя законов, институтов, правил и традиций, установленных нашими предками, чтобы правильно и адекватно выражать в своё время определённые принципы, которые в существующей форме больше не выражают эти идеалы сегодня.

Нежелание признавать необходимость перемен глубоко укоренилось в человеческой природе. Большинство людей считают, что обсуждение политических изменений преждевременно — пока оно не устарело. Любое планирование будущего, любое предвидение и ясное видение грядущих перемен они называли <<утопическими>>.

Они не понимают, что критерием утопии всегда является попытка организовать настоящее или будущее в соответствии с образом прошлого. Все те мечтатели прошлых веков, которых мы сегодня называем утопистами, мечтали о будущем с пасторальной атмосферой. Каждый проповедовал своего рода <<возвращение к природе>>.

Те, кто хочет только <<защищать>> и <<консервировать>>, не осознают, что на самом деле они сами являются утопистами, потому что верят в возможность вечно поддерживать определённые формы и институты, не приспосабливая их к постоянным изменениям этого мира.

Итак, сегодня мы видим человека, который хочет восстановить старую Римскую империю 2000-летней давности. Другой человек мечтает о восстановлении Священной Римской империи Германской нации, умершей так много веков назад. И эти люди притворяются лидерами молодёжи, пророками будущего, создателями нового порядка. Они осмеливаются осуждать идеалы демократии и свободы личности, существующие едва ли 150 лет, как избитые и устаревшие.

Не только допотопные диктатуры, но и демократические народы управляются сегодня утопистами высшей степени, людьми, которые поклоняются <<нации>> и <<расе>>, которые считают себя способными создать Лигу наций, состоящую из суверенных национальных государств, которые верят в производство материальных благ при помощи тарифных барьеров и автаркии, которые не хотят <<вмешиваться>> во внутренние дела других государств, которые хотят жить <<изолированно>> и чтобы их <<оставили в покое>>.

Они пытаются убедить нас в том, что применение демократических идей в ведении международных дел потерпело неудачу, что свобода торговли потерпела неудачу, что Лига Наций потерпела неудачу, что Англо-германское морское соглашение (1935) и Пакт Келлога [\textit{автор.} Naval and Kellogg Pacts] потерпели неудачу, и что, следовательно, нет никаких шансов добиться принятия демократических принципов в международных отношениях. И они приходят к выводу, что мы должны ограничить демократию во внутренних делах и свести к минимуму контакты с внешним миром.

Нелепо притворяться, что из-за того, что первая в истории попытка организовать международную жизнь на демократической основе провалилась, поскольку были нарушены Устав Лиги и другие соглашения, то принципы, которые они пытались выразить, оказались ошибочными и, следовательно, должны быть упразднены.

На заре человеческой цивилизации было сформировано несколько примитивных правил, которые рассматривались как \textit{непременное условие} [\textit{conditio sine qua non}] любой общественной жизни и которые назывались Десятью заповедями. На протяжении двадцати столетий самые могущественные организации, представленные Церковью и государствами, делали всё возможное, чтобы эти примитивные принципы Десяти Заповедей соблюдались повсеместно. Чтобы осуществить эту цель, государства и Церковь имели в своём распоряжении такие средства, как эшафот, полиция, анафема, армии, дьявол, тюрьма и ад. Несмотря на все эти разнообразные материальные и духовные меры принуждения, и по прошествии двадцати столетий в самых высокоцивилизованных и самых христианских странах мира не проходит и дня, чтобы не происходили убийства, кражи и прелюбодеяния.

Cтал бы кто-нибудь утверждать, что из-за того, что убийцы и воры всё ещё существуют после стольких столетий попыток покончить с ними, Десять заповедей оказались бесполезными, неприменимыми и, следовательно, должны быть упразднены?

Но есть огромное количество людей, которые выдают себя за государственных деятелей и которые серьёзно заявляют, что поскольку первая в истории человечества попытка организовать международную жизнь по определённым принципам провалилась, эти принципы неприменимы, неэффективны и должны быть отброшены.

Сегодня практически не публикуется политических статей, которые не начинались бы следующими фразами: <<Техническое развитие делает мир всё меньше и меньше. Расстояния исчезают, и люди и народы живут всё ближе и ближе друг к другу. Такое развитие техники превращает тарифные барьеры, национальные антагонизмы и войны в анахронизм и автоматически делает невозможным ведение войн>>.

Этот вывод является не полной правдой. То, что эволюция техники сближает людей и континенты — это верно, но такое географическое и физическое <<сближение>> может иметь два последствия: (1) политическое и экономическое сближение или (2) схватки и разногласия, более разрушительные, чем когда-либо, именно из-за тесного соседства людей друг с другом. Какой из этих двух вариантов возникнет, зависит от вопросов по сути нетехнического характера.

Ещё одна причина, почему перемены, через которые мы проходим, столь болезненны, заключается в том, что новаторский период индустриализма закончился. Производство текстиля или строительство железных дорог больше не является большим личным достижением. Сто или даже пятьдесят лет назад эти начинания требовали большого личного риска, дерзкой личной инициативы, изобретательности, храбрости и решимости бороться. Но сегодня промышленная деятельность превратилась в более или менее рутинную, административную работу.

Мы видели, как Россия за двадцать лет построила промышленную организацию, на создание которой Англии, Соединённым Штатам и Германии потребовалось более столетия. Это <<чудо>>, несомненно, вскоре повторится в Индии и Китае. Мы не можем остановить эту эволюцию и не можем пытаться искусственными средствами сосредоточить промышленную мощь в руках определённых людей, определённых корпораций или определённых наций. Такая попытка почти наверняка приведёт к взрывам.

Статичная концепция поддержания статус-кво — любого статус-кво — и примитивные инстинкты тех наций, которые не удовлетворены данным статус-кво (а такие неудовлетворенные нации всегда будут) и вносят изменения силой — это не две разные политики, а две стороны одной и той же политической концепции.

И та, и другая ложны. Обе политики должны приводить от одной войны к другой, потому что они не делают ничего, кроме как увековечивают тенденцию, которой человечество следует с незапамятных времён.

При планировании лучшего ближайшего будущего крайне важно предотвратить формирование таких жёстких конструкций, которые через одно или два поколения могут стать такими же реакционными и устаревшими, как большинство наших нынешних институтов.

Те, кто отказался от идеалов экономической свободы, в течение некоторого времени пропагандировали плановую экономику, контролируемую государством. Благодаря таким различным национальным плановым экономикам экономическая конкуренция и анархия стали только хуже, чем когда-либо, потому что суверенные единицы, участвующие в этой конкуренции, намного могущественнее, чем были отдельные индивидуумы. Совершенно очевидно, что ни одна экономическая проблема не может быть удовлетворительно решена, если планирование будет оставаться суверенной компетенцией \textit{многочисленных} национальных правительств.

Ещё более важно предотвратить формирование таких жёстких политических конструкций, которые только усугубили бы и расширили будущие конфликты. В качестве средства достижения цели региональные решения могут быть незаменимы, но было бы опасно полагать, что такие решения, как Пан-Европа, Пан-Азия или Пан-Америка, решат наши международные проблемы на более чем очень короткое время.

Что могло бы произойти, если гипотетически предположить то, что мы могли бы организовать мир на пяти или шести таких крупных политических единицах, не меняя фундаментальных принципов международных отношений, мы можем ясно видеть в работах сторонника Панъевропейского союза.

Граф Куденхове-Калерги в своей книге \textit{<<Европа просыпается!>>}\footnote{%
\textit{Coudenhove-Kalergi: Europa Erwacht, page 178. — Прим. автора.}} пишет следующее: <<Европа должна при любых условиях отказаться от роли мирового полицейского. Международная армия или военно-воздушные силы, охватывающие и другие континенты, будут больше угрожать европейскому миру, чем защищать его. Китайский вопрос может привести в ближайшие годы и десятилетия к величайшим конфликтам в Восточной Азии, которые Европа не сможет предотвратить, но в которые Европа будет втянута, если не сохранит строгий нейтралитет\ldots\ Пан-Европа требует европейской армии для защиты Европы и обеспечения безопасности европейского мира, но она отвергает армию Лиги, которая не гарантировала бы этот мир, но которая угрожала бы ему>>.

Это дословное повторение аргументов самых ярых изоляционистов, сторонников невмешательства и нейтралитета, с той лишь разницей, что это заявлено от имени территории, несколько большей, чем Франция, Германия или Испания, но меньшей, чем Соединённые Штаты, Бразилия или Россия. Трудно понять, почему панъевропейская политика невмешательства в Китай была бы более мудрой, чем англо-французская политика невмешательства в Испании; почему строго панъевропейская армия лучше защитила бы Европу, чем строго американская армия была способна предотвратить нападение на Соединённые Штаты, и почему <<строгий нейтралитет>> Пан-Европы в предполагаемом дальневосточном конфликте был бы более высокой моральной позицией, с точки зрения международной жизни, чем <<нейтралитет>> Франции и Англии проявленный во время чехословацкого кризиса.

В колоссальных политических структурах нет никакого спасения в отсутствие разъяснения и безошибочной интерпретации тех первичных принципов, на которых должна основываться \textit{любая} форма международного порядка.

Пренебрежение этой элементарной необходимостью также стало причиной провала Лиги Наций. Устав был, по сути, международным документом, но его подписантами были суверенные национальные государства. В здании Лиги Наций царила интернациональная атмосфера. Лига была своего рода аристократическим клубом, где периодически собирались одни и те же несколько сотен государственных деятелей и те же несколько сотен газетчиков. Было замечательно, что у них было место встречи, что они могли обменяться своими мнениями. Но на родине ничего не изменилось. В каждой стране всё образование продолжало носить националистический характер. Пресса оставалась националистической. Администрация и правительство оставались националистическими.

Было наивной утопией полагать, что представители и функционеры таких националистических государств могли бы путём периодических встреч добиться лучшего международного взаимопонимания. Это было всё равно что начинать строительство нового здания с крыши. Есть только один способ строить, и это начинать с фундамента, даже если строительство таким способом потребует более длительной работы.

В политике очень важно отличать возможное от невозможного, реальность от утопии. Утопию никогда нельзя охарактеризовать по её существу. Двумя характерными чертами любой концепции утопии являются (1) желание спроецировать прошлое в будущее; (2) вера в то, что технические достижения могут сделать человеческую природу лучше и добрее.

Мы всё ещё далеки от понимания точного научного механизма социальной и экономической жизни. Но эксперименты последних двадцати лет достаточно убедительно доказывают, что политика недалека от математики, и что у нас нет никаких оснований ожидать, что если мы будем достаточно долго стараться, то сможем получить пять, сложив два и два.
