\chapter{Невмешательство}

Доктрина, которая, возможно, вызвала наибольший хаос в международных отношениях, — это принцип невмешательства. Этот принцип, который настолько глубоко укоренился в умах наших государственных деятелей и дипломатов, что его можно назвать догмой, находится в таком полном противоречии со всеми проявлениями современной жизни наций, что его последствия за последние двадцать лет были катастрофическими. Эта доктрина была одной из главных причин, по которой банда беспринципных гангстеров смогла добиться высочайшей власти в Европе.
 
Принцип <<невмешательства во внутренние дела других стран>> был установлен столетия назад потомственными монархиями. Это был вопрос учтивости между джентльменами, выражающий те же чувства, которые испытывают многие родственные семейства по отношению к домашним делам друг друга. Народами правили давние исторические династии. Решения о войне и мире принимались монархами, и между великими правящими домами существовало взаимопонимание, что они не должны вмешиваться во внутренние дела друг друга. В то время это был оправданный принцип.

В период перехода от абсолютных монархий к демократическим нациям эта идея была унаследована, поскольку в интересах новообразованных демократий было, чтобы оставшиеся короли не вмешивались в их внутреннюю конституционную жизнь. Но по мере того, как шло время и развитие промышленности, торговли и коммуникаций превращало весь мир в единую экономическую единицу, этот принцип невмешательства во внутренние дела других стран превратился в фарс.
 
Он оказывает такое же влияние на нашу международную жизнь, какое <<сухой закон>> оказал на общественную жизнь Соединённых Штатов. Будучи полностью противоречащим реальности, он не только не защищает нацию от иностранного вмешательства, но и делает её предметом всех видов незаконного вмешательства.

Нет необходимости перечислять все случаи, когда, прикрываясь лицемерным невмешательством, тоталитарные державы вмешивались во внутренние дела других стран, создавали свои собственные организации, подрывали существующий социальный порядок, подкупали и коррумпировали людей и институты, провоцировали и одобряли убийства, революции и гражданские войны.
 
Едва ли какая-либо нация на этой планете была избавлена от этих внутренних потрясений. Но демократические страны закрыли глаза на эти факты, заявив, что это не их дело, что они ничего не могут сделать, потому что любое действие противоречило бы принципу невмешательства. И этот принцип был неприкосновенен.

Но помимо огромного числа катастроф, которые были прямым результатом <<невмешательства>>, каковы возможности политики, основанной на этом принципе, и каково реальное отношение внутренних дел наций друг к другу?
 
Давайте посмотрим на мир таким, каким он был создан и организован после Первой мировой войны. Те, кто создавал этот порядок, верили, что они заложили основу для более тесных международных отношений, для лучшего распределения богатств, для расширения международной торговли, для разоружения, процветания и постоянного прогресса. И что же произошло?

Вновь созданные государства начинали с политики <<самодостаточности>>. Чехословакия, которая унаследовала крупные промышленные предприятия бывшей австро-венгерской монархии, была преимущественно индустриальной страной. Венгрия в том виде, в каком она пребывала, была чисто сельскохозяйственной страной. В течение многих лет чехословацкое правительство проводило одностороннюю сельскохозяйственную политику, создавая в основном искусственными средствами и субсидиями внутреннее сельское хозяйство, чтобы производить необходимые продукты питания внутри страны, в результате чего Венгрия и другие балканские страны потеряли естественный рынок сбыта своих сельскохозяйственных излишков и не смогли закупать промышленную продукцию из Чехословакии и других промышленно развитых стран. В качестве <<единственного лекарства>> они начали создавать, в равной степени с помощью чисто искусственных средств и государственных субсидий, новую отрасль промышленности, чтобы производить у себя все продукты, в которых нуждались их фермеры. Результатом стал рост цен и снижение уровня жизни во всех соответствующих странах.
 
Когда в результате мирового экономического кризиса Британия отказалась от золотого стандарта и девальвировала фунт стерлингов, Соединённые Штаты вскоре после этого были вынуждены пропорционально девальвировать доллар.

Когда началась Конференция по разоружению и демократические страны были готовы и желали сократить свои вооружения, итальянский милитаризм и усиление вооружений Германии сорвали все попытки достичь соглашения и вынудили демократические страны также перевооружиться.
 
Примеры можно было бы множить почти без ограничений. В каждом случае мы видим, что внутренняя политика определённого правительства или правительств — экономическая или финансовая, в трудовой или военной сфере — вынуждала другие правительства адаптировать свою собственную политику к новым условиям, созданным другими.

Богатейшая страна мира — Соединённые Штаты Америки, изобилующая производством потребительских товаров и где запасы сырья казались неисчерпаемыми, была вынуждена прекратить производство автомобилей для частного потребления, ввести нормы потребления сахара, мобилизовать всю свою рабочую силу, ограничить личные доходы, обложить налогом всю сверхприбыль, тем самым изменив самым радикальным образом американский образ жизни.
 
Почему?

Эти революционные перемены в могущественных Соединённых Штатах Америки, столь уверенных в своей мощи, в своём политическом и экономическом превосходстве и в своей отстраненности от <<зарубежных споров>>, являются \textit{прямыми последствиями} того факта, что за несколько лет до этого правительство Германии прекратило производство автомобилей для частного использования, призывала на военную службу свою молодёжь, нормировала сахар, облагала налогом сверхприбыль и ограничивала доходы физических лиц. Тем не менее, в Соединённых Штатах есть люди, которые, став свидетелями того, как внутренняя политика страны, находящейся за 4000 миль от них, непосредственно и глубоко повлияла на внутреннюю политику Соединённых Штатов и повседневную жизнь их граждан, всё ещё серьёзно говорят о невмешательстве как о <<политике>>!
 
Если одна нация проводит автаркическую политику, это нарушает всю международную торговлю и влияет на уровень жизни каждого человека в других нациях. Если одна нация вступает в гонку вооружений, то любая другая нация вынуждена тратить большую часть своего национального дохода на вооружение. Если одна нация рассматривает договор как клочок бумаги, договоры в целом теряют свою ценность. Если одна значительная нация девальвирует свою валюту по внутренним финансовым причинам, другие страны вынуждены последовать её примеру. Если одно правительство принуждает своих рабочих работать по шестьдесят часов в неделю, оно реагирует на условия труда в других странах. Если одна нация начинает воспитывать свою молодёжь на чисто милитаристской основе, все остальные нации должны готовиться к военной организации. Если одно правительство без зазрения совести использует преимущества диктатуры, все остальные страны вынуждены отказываться от всё большего и большего числа своих демократических достижений.

Внутренняя жизнь отдельных наций настолько переплетена, их влияние друг на друга настолько очевидно, что этот взаимообмен причин и следствий является одной из немногих кристаллизаций международной жизни, обладающей характеристиками социального закона.
 
И этот закон, по-видимому, очень своеобразен. В некоторых отношениях он напоминает теорию предельной полезности в экономике, согласно которой цена предмета потребления определяется затратами производства для производителя, работающего в наименее благоприятных условиях среди тех, кто способен конкурировать.

Похоже, что в современной взаимосвязанной жизни наций уровень жизни, культура, условия труда, индивидуальная свобода, налогообложение, экспортная торговля, оборонная политика — всё это в значительной степени зависит от политики, проводимой в тех же областях другими нациями. И определяющим фактором является нация, живущая в самых низких моральных и наименее благоприятных экономических условиях.
 
Похоже, что на данной территории — безусловно, на европейском континенте, но, возможно, и на всем земном шаре — существует какой-то закон или, по крайней мере, какая-то тенденция, согласно которой среди определённого числа взаимозависимых наций все государства вынуждены приспосабливать свою форму правления к наименее цивилизованной форме правления, существующей в любом из этих государств.
