\chapter{Нейтральность}

На панамериканской конференции в Рио-де-Жанейро господин Самнер Уэллес, представитель Соединённых Штатов Америки, выступая перед двадцатью одной Южноамериканской республикой 15 января 1942 года, сделал следующее заявление:
 
<<История последних двух лет неизменно перед нами. Мы с вами знаем, что если бы в течение последнего десятилетия существовал международный порядок, основанный на законе, и были бы возможности обеспечить соблюдение такого закона, земля сегодня не подвергалась бы жестокому бедствию, которое сейчас опустошает весь земной шар. Если бы законопослушные и мирные народы Европы были готовы объединиться, когда угроза гитлеризма впервые начала проявляться, то Гитлер никогда бы не осмелился встать на свой пагубный путь. Исключительно из-за того факта, что эти нации, вместо того чтобы сплотиться, позволили себе держаться в стороне друг от друга и возлагали надежду на спасение за счёт своего собственного нейтралитета, Гитлеру удалось захватить их одну за другой, когда время и обстоятельства сделали это целесообразным>>.

Представитель правительства Соединённых Штатов заявил далее:

<<Принцип классического нейтралитета в его узком смысле в этом трагическом современном мире больше не может быть каким-либо реальным нейтралитетом между силами зла и силами, которые борются за сохранение прав и независимости свободных народов. Гораздо лучше для любого народа славно бороться за защиту своей независимости; гораздо лучше для любого народа умереть, если потребуется, в битве за спасение своих свобод, чем, цепляясь за потрепанную фикцию иллюзорного нейтралитета, преуспеть только в том, чтобы совершить самоубийство>>.

Эти слова, произнесённые представителем Соединённых Штатов Америки, являются самым трагическим самоосуждением собственной прежней политики, когда-либо произнесённым правительством.

Это были именно те слова, с помощью которых Великобритания и Франция пытались убедить Соединённые Штаты Америки в тщеславии, неэффективности и бессодержательности их политики нейтралитета в течение последних лет. Это были те самые слова, которые испанское, австрийское и эфиопское правительства тщетно повторяли снова и снова британскому и французскому правительствам. Это были те самые слова, которые не были услышаны и не возымели никакого эффекта, когда чехи адресовали их полякам, поляки — скандинавам, греки — югославам, а югославы — туркам.

Сегодня, когда <<политика нейтралитета>> является катастрофической неудачей по всему земному шару, когда ни одна нейтральная нация — за исключением нескольких второстепенных стран — не избежала агрессии, войны и завоевания благодаря своему нейтралитету, уже не будет великой государственной мудростью признать, что нейтралитет — это самоубийство.

Факт остаётся фактом, что ни одна нация не отказалась от своего нейтралитета в этом мировом конфликте, пока её не принудили к этому; что ни одна нация не допустила какой-либо критики своей политики нейтралитета и что ни одна демократическая держава не осмелилась притронуться к догме нейтралитета, даже \textit{во время} войны, когда во многих случаях было очевидно, что провозглашение или признание нейтралитета являлось самым верным путём к военному поражению. Таким образом, мы стали свидетелями не только маньчжурско-эфиопско-испанского фарса во времена <<мира>>, но и трагикомедии высадки британских войск в греческих портах на глазах у немецких дипломатов и работы британской ближневосточной штаб-квартиры в Каире под постоянным наблюдением дипломатических представителей стран <<оси>> на месте.

Активная помощь, оказываемая диктаторским державам в этой войне посредством принципа нейтралитета, неизмерима.

Ряд пактов о ненападении и гарантиях нейтралитета, подписанных и провозглашённых Гитлером в последние годы, таким образом, позволили завоевать все страны, которые он никогда не смог бы завоевать, если бы они были объединены, но которые он поглощал одну за другой, были величайшим триумфом обольщения со времён Казановы и Дон Жуана. Жертвы были выставлены на посмешище тем более, что каждый из них, от крошечного Люксембурга до могущественных Соединённых Штатов, глубоко верил, что к нему будут относиться по-другому и что слащавые слова Гитлера, обращенные к нему, в его особом случае были действительно искренними.

Догматическая вера в нейтралитет настолько сильна, что даже сегодня, после завоевания стольких стран исключительно благодаря их нейтралитету, многие представители этих стран, поляки и голландцы, румыны и норвежцы, защищают свою старую бесполезную политику, утверждая, что нейтралитет сам по себе был правильным, и они не могли следовать никакой другой политике, кроме той, которой они следовали.

Поскольку маловероятно, что люди извлекут уроки из опыта, каким бы трагичным и каким бы близким он им ни был, почти наверняка они снова вернутся к этой политике, если только мы не осознаем полную историческую, моральную и политическую неосуществимость принципа нейтралитета.

Идея нейтралитета восходит к тем эпохам, когда войны признавались правовыми актами политики, когда войны велись между династиями, между странами за изменение границ и колониальную экспансию. В те периоды, когда войны обычно ограничивались двумя или несколькими участниками, когда они никогда не выходили за пределы определённых географических границ, когда они велись в соответствии с определёнными принятыми правилами и обычно заканчивались переговорами и примирением, было возможно установить принцип, согласно которому определённые войны не затрагивали страны за пределами конфликта или его возможной опасности.

Но поскольку технический прогресс и промышленная эволюция превратили землю в маленькую часть; поскольку войны больше не ведутся из-за династической ревности или местных разногласий; поскольку мы ведем великую борьбу за власть, чтобы поставить землю или, по крайней мере, основные её части под единый контроль; поскольку демократические страны пытаются организовать эту борьбу за мировую власть без кровопролития и пытаются объявить войны вне закона, называя их преступлением, нейтралитет — это не только политическая бессмыслица, но и величайшая моральная низость.

Расценивать определённые действия тоталитарных режимов как преступные и заявлять о намерении сохранять нейтралитет перед разворачивающейся драмой — полное противоречие.

Любое заявление о нейтралитете одобряет действия фашистских, нацистских и японских милитаристов, поскольку оно отображает неявно допущение, что нарушения международной морали, совершаемые тоталитарными силами, допускают позицию нейтралитета.

Что действительно отвратительно в трагических событиях последних десяти лет, так это не повторяющееся злоупотребление доверием, предательство, агрессия, жестокие завоевания и порабощение сотен миллионов людей тоталитарными правительствами, а отношение тех людей в цивилизованных христианских странах, которые обладают моральной способностью признать, что мы сталкиваемся с самыми возмутительными преступлениями, когда-либо совершавшимися на этой земле в любой форме, и которые, несмотря на осознание правды, считают себя вправе сказать: нас это не касается; мы хотим оставаться в стороне.

Каким бы ни было оправдание такому отношению, будь то страх, трусость, безразличие или слепота, с этической точки зрения не подлежит сомнению, что те, у кого хватает моральных сил признать преступление и всё же мириться с ним, более виновны, чем те, кто его совершает.

