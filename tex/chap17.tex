\chapter{Принципы и институты}

Мы всегда должны иметь в виду, что любые теоретические системы человеческого общества не только неосуществимы и никогда не существовали, но и \textit{в целом} [\textit{in toto}] представляют собой невозможность, такое положение дел, при котором было бы невыносимо жить. Реальная жизнь в любой момент истории всегда была <<переходным периодом>>, содержащим множество элементов различных систем прошлого. Ни одно радикальное изменение, ни одна революция никогда не были способны достичь большего, чем часть своей программы, из которой жизнь синтезировала многие элементы прошлого.
 
Что требуется, так это не детализированная и сама по себе закрытая система будущего, а чёткое понимание основных принципов, которые указывают нам дорогу, ведущую в верном направлении.

Будет ли следующим шагом новая Лига Наций, региональные группировки, континентальные организации, союз англоязычного мира или мировое правительство, имеет второстепенное значение. Существенным фактом является то, что мы понимаем, что означают принципы демократии в терминах двадцатого века, и что на их основе должен начаться процесс \textit{международной интеграции}.
 
Следуя этой естественной эволюции, любая из вышеупомянутых конструкций помогла бы нам сделать большой шаг вперед.

В какой-то международной хартии мы должны подтвердить принципы Великой хартии вольностей, Декларации независимости, Билля о правах и Декларации прав человека и гражданина. Каковы эти основные принципы демократии? И каково их значение в середине двадцатого века?
 
\textit{Первое:} Право на свободу. В первоначальных документах было чётко определено, что свобода заключается в <<способности делать всё, что не ущемляет свободу другого>>. В переводе на международный язык это означает, что каждая нация должна быть свободной и независимой, но только в той мере, в какой возможность осуществления этого права не наносит ущерба свободе и независимости других наций. В настоящее время эти ограничения национальной независимости и национальной свободы не существуют и не определены, а без такого определения свобода и независимость наций бессмысленны. Они приводят только к войнам.

\textit{Второе:} Согласно первоначальным хартиям демократии, равенство означает, что закон должен применяться в равной мере к каждому индивидууму, независимо от того, защищает ли он человека или наказывает его. Переведённый сегодня на международный уровень, этот принцип означает, что все нации должны быть равны перед законом. При том, как устроен мир сегодня, международного права вообще не существует, а без такого закона <<равенство>> наций бессмысленно и ведёт только к войнам.
 
\textit{Третье:} Первоначальные хартии демократии гарантируют каждому человеку право на безопасность и утверждают, что безопасность является результатом сотрудничества всех для обеспечения прав каждого. В отношении международного применения это означает, что безопасность каждой нации может быть результатом только сотрудничества всех других наций для обеспечения прав каждой из них. Это определение явно ставит вне закона такие концепции, как нейтралитет или невмешательство, которые никогда не смогут обеспечить безопасность ни одной нации и которые противоречат сути основных хартий демократических принципов. Мы видели, как пренебрежение первоначальной концепцией безопасности, как нейтралитет и невмешательство привели каждую нацию, которая верила в них, к войнам и разрушениям.

\textit{Четвертое:} В первоначальных хартиях демократии говорится, что суверенитет, по сути, принадлежит сообществу — всеобщности граждан. В них прямо говорится, что ни один человек и ни одна группа лиц не могут осуществлять суверенную власть. Как мир организован сегодня, очевидно, что суверенитет заключается не во всеобщности граждан, а в том, что вопреки духу первоначальных хартий демократии суверенные права осуществляются группами индивидуумов, которых мы называем <<государствами>>.
 
Существование нескольких сотен суверенных государств, осуществляющих суверенную власть, находится в полном противоречии с демократической концепцией суверенитета, которая должна основываться на сообществе. Поэтому крайне важно осуществить разделение властей в этой области, вернуть сообществу абсолютный суверенитет и предоставить отдельным нациям и отдельным государствам только такую суверенную власть, которая имеет своим источником всеобщий суверенитет.

Это первые шаги на пути к конституционной жизни в мировых делах. В Декларации прав человека и гражданина говорится: <<Любое общество, в котором не обеспечены гарантии прав или не определено разделение властей, вообще не имеет конституции>>.
 
Что касается обязанностей отдельных наций друг перед другом и перед сообществом, то это также было ясно выражено в великих демократических хартиях мыслью о том, что любой, кто нарушает закон, объявляет себя находящимся в состоянии войны с обществом.

Эти принципы должны быть разъяснены и систематизированы \textit{сейчас}, во время этой войны. Они должны быть написаны золотыми буквами на наших флагах; они должны стать душой наших солдат; они должны быть на всех длинах волн.
 
Наша победа должна стать победой этих новых принципов, на основе которых мы сможем построить новое и лучшее мировое общество. Эти принципы и обещание того образа жизни, который они нам предлагают, являются нашим самым мощным оружием. Это единственное оружие, которое может придать достаточную огневую мощь нашим бомбардировщикам, танкам и военным кораблям.

Провозглашение этих принципов нельзя откладывать, они будут обсуждаться после \textit{победы}. Они — крылья, они — единственное историческое оправдание нашей грядущей победы.
 
В истории всегда во время войн происходили крупные революционные перемены. Мы должны сформировать аморфные массы на пяти континентах сейчас, прежде чем они снова затвердеют в форме, противоречащей нашим представлениям.

Мы должны действовать сейчас, во время нынешней войны, потому что никакая военная победа не может дать нам гарантии того, что она приведёт к созданию разумного мира. Только политическая победа может сделать это. И мы никогда не сможем одержать политическую победу, не ведя политическую войну одновременно с военной войной.
 
Одной из величайших трагедий нашего времени было то, что демократические нации и демократические правительства не осознавали — и до сих пор не осознают — того факта, что идёт гигантская политическая борьба, одним из симптомов которой является военная война.

Мы должны возглавить человечество в соответствии с историческими потребностями, иначе мы потеряем лидерство. Сейчас мы находимся в эпицентре политической и социальной революции, в которой международная война является лишь одной частью. Мы должны быть крестоносцами новых идей. Мы не должны продолжать оставаться защитниками бесполезных систем прошлого, восстановление которых на протяжении всей истории всегда доказывало свою утопичность.
 
Отмена международного и экономического партикуляризма является исторической необходимостью. Результатом этой войны станет ограничение национальных суверенитетов и начало процесса международной интеграции.

Эти события могут происходить в двух формах: либо по взаимному согласию между доселе независимыми и суверенными нациями, либо путём насильственного навязывания.
 
Если новый демократический порядок должен быть создан принудительным путём — а согласно историческим прецедентам, скорее всего, так и будет, — тогда важно, чтобы англо-американские нации взялись за эту задачу. Они должны предпринять это не только потому, что от надлежащего переустройства мира будет зависеть выживание их собственных демократических институтов и само существование их народов, но и потому, что последние несколько столетий доказали, что на нынешнем этапе истории человечества англо-американское превосходство означает общий прогресс для всего человечества, тогда как все попытки доминирования со стороны любой другой потенциальной мировой державы всегда означали противодействие в отношении к демократической эволюции.

Демократические страны должны отказаться от своих статичных и оборонительных концепций и проникнуться динамичным духом атаки и завоевания. Только идеалы и принципы могут добиться этого.
 
У нас не будет победы, если мы хотим только, чтобы нас оставили в покое и чтобы мы защищали то, чем обладаем. Если демократические народы действительно ценят свои принципы, дорожат своей свободой и преданы своему образу жизни, они не могут принять или мириться с установлением по соседству с ними политических и социальных концепций, представляющих собой полное отрицание их собственных принципов. Они должны обладать железной волей, чтобы распространять свои идеи по всему миру и бороться с врагами своих концепций и идеалов, где бы они ни находились. Мы не можем выиграть эту войну за демократию без убеждённости.

И есть только один критерий подлинной убеждённости — желание распространять её.
 
Если мы обеспечим соблюдение законов, гарантирующих свободу человека повсюду, и разъясним, что это за свободы, предусмотрев в одних и тех же законах ограничения и защиту этой свободы; если мы сможем разъяснить, с помощью каких ограничений мы можем получить реальную свободу слова, свободу печати и свободу собраний; если мы соблюдаем принципы международных отношений, которые идентичны принципам взаимоотношений между отдельными людьми в демократическом государстве; если мы провозгласим взаимозависимость наций, ограничим национальный суверенитет, объявим нейтралитет вне закона и создадим организацию, обладающую силой для защиты этих принципов, для судебного преследования и наказания любой нации, нарушающей установленные законы и принципы, — тогда не будет иметь большого значения, какие внешние формы принимают нации или группы наций.

Наша цель не должна быть в том, чтобы уничтожить разнообразие в этом мире. Культурное и традиционное разнообразие, существующее между разными нациями, является величайшим шармом нашего существования. Наша цель должна заключаться лишь в том, чтобы не допустить вырождения этого разнообразия в вооружённых конфликтах, направить вечную борьбу за жизнь в более цивилизованное русло и создать политический порядок, который, наконец, сделает возможным решение экономических и социальных проблем нашего \textit{века}.
 
Это выглядит так, как если бы политическое единство было единственной возможностью сохранить и обезопасить культурное разнообразие.

Исход этой войны и развитие событий в наступающем столетии в значительной степени будут зависеть от того, кем и при каких условиях будут заложены эти основы обновлённого демократического мира. Большим препятствием для мирных и свободолюбивых наций является то, что сегодня ими руководит правящий класс, полностью лишённый дальновидности, таланта, силы воли и способности к действию. Весьма сомнительно, что те люди, которые потеряли мир и кажутся неспособными понять истинное значение и характер этой мировой войны, смогут создать новый мировой порядок, превосходящий прошлый.
 
Отбор лидеров — важнейшая проблема любой демократической организации. Как этот сложный механизм работает сегодня, кажется, что качество государственного управления, дальновидность, мудрое лидерство, самопожертвование и способность к действию существенно отличаются от качеств, необходимых для получения власти.

Вопрос о персонале чрезвычайно важен, поскольку человек — это начало и конец социальной жизни, и, в конечном счёте, каждая идея, учреждение и административная должность представлены людьми. С самого зарождения демократических обществ идеалом было то, что самый способный человек, независимо от ранга, богатства и происхождения, должен найти свой путь к вершине. Предполагалось, что всеобщее голосование станет правильным путём, ведущим к этой цели.
 
Если мы изучим досье тех людей, которые пришли к власти таким образом и которые имели влиятельный голос в формировании общественной жизни в течение последних лет, у нас могут возникнуть обоснованные сомнения в том, является ли нынешний метод выдвижения кандидатов и избрания в законодательные органы и правительства достаточно избирательным, чтобы гарантировать наилучшее возможное лидерство. Существует слишком много случаев, когда вскоре становилось очевидным, что тем, на кого была возложена ответственность за формулирование и управление нашими делами, не хватало элементарных знаний о проблемах, с которыми предстояло иметь дело, и они не обладали основными качествами лидера.

В любой другой области человеческой деятельности для любого продвижения необходимы определённая способность к рассуждению и определённое элементарное знание фактов. Среди астрономов могут существовать сотни различных взглядов на устройство Вселенной. Все эти расходящиеся точки зрения должны свободно и тщательно обсуждаться учеными в университетах и академиях. Только посредством такого свободного обсуждения может быть принята истинная и точная теория и определены надлежащие авторитеты. Но если бы кто-нибудь в таких дискуссиях утверждал и упорствовал в мысли, что земля — это не шар, а плоская тарелка, окружённая водой, и что она не вращается вокруг солнца, а что солнце вращается вокруг земли, ему не разрешили бы преподавать в университетах, он бы не получал академических наград, и уж точно не считался бы научным авторитетом.
 
Никому не покажется, что отстранение такого явно неквалифицированного человека от должности профессора в университете противоречит свободе науки и антидемократично. Это просто общепринятый факт, что дискуссия, которая всегда должна оставаться свободной ради научной истины и прогресса, выше тех границ, которые он старается сохранить и чему учит, и что никого, придерживающегося таких взглядов, нельзя воспринимать всерьёз.

Существуют тысячи проблем, касающихся строения человеческого организма, и эти проблемы должны свободно обсуждаться во всех медицинских кругах. Каждое мнение должно быть выражено и изучено с предельной тщательностью и вниманием. Только таким образом лучшие умы медицинской науки смогут выйти на передний план и конструктивно помочь в борьбе с болезнями. Но если бы в таких медицинских дискуссиях кто-то стал утверждать и стойко отстаивать своё утверждение о том, что в человеческом теле не существует такой вещи, как кровообращение, никто ни на одном собрании представителей медицинской науки не стал бы его слушать. Вполне естественно, что на него будут смотреть как на невежду. И он тщетно настаивал бы на том, чтобы ему разрешили стать профессором медицины на том основании, что свобода слова и свобода науки дают ему право проповедовать эти взгляды.
 
Эти принципы отбора универсальны во всех сферах человеческой деятельности, за исключением политической.

Однако здесь проблемы, с которыми приходится иметь дело, являются самыми сложными и существенно влияют на саму жизнь сотен миллионов, мы всё ещё прислушиваемся к людям, которые считают, что такие понятия, как нейтралитет, изоляция, невмешательство и т.д. и т.п., могли бы стать предметом серьёзного общественного обсуждения. Это такие же мёртвые концепции, как представления о земном шаре до Коперника и теории о крови до Гарвея.
 
Мы должны скорректировать нашу нынешнюю систему избрания представителей и правительства и должны требовать, чтобы тот, кого мы посылаем в законодательное собрание, обладал не только яркой личностью, влиятельными друзьями и ораторскими талантами, но и определённым минимумом знаний об общественных делах и демократических принципах.

Исправление и ограничение таким образом нашей нынешней системы свободных выборов не только не нарушило бы демократический принцип свободного голосования, но и дало бы гораздо лучшую возможность людям осуществить своё демократическое право на избрание, доверив представительство таким людям, которые действительно представляют демократические идеи.
 
Критерием демократии не может быть то, что человек с антидемократическими идеями должен иметь возможность быть избранным. Только люди с правильным представлением о современном мире и с достаточно глубоко укоренившимися демократическими убеждениями могут привести нас к следующему шагу: к основанию демократической международной жизни.

Скептики, очевидно, спросят, кто будет судьей, кто гарантирует, что такими ограничениями свободы слова, свободы собраний, свободы прессы злоупотреблять не будут. И кто будет судить о том, обладают ли люди, которые будут избраны, необходимыми качествами для демократического руководства. Ответ прост:
 
Время от времени, естественно, будут происходить злоупотребления, поскольку не существует возможной организации общества, совершенной в абсолютном смысле.

Во многих случаях единственным возможным способом вылечить смертельную болезнь является применение терапевтических мер, которые, преодолевая недуг, в большей или меньшей степени воздействуют на другие органы тела, не затронутые болезнью. Тем не менее, это единственный признанный метод лечения, который не оспаривается никакими органами власти. Никто также не оспаривает срочную необходимость хирургических операций, несмотря на полное осознание рисков, с которыми они могут быть связаны.
 
Мы пришли к заключению, что нынешняя интерпретация демократических принципов и некоторых демократических институтов в том виде, в каком они функционируют сегодня, представляет смертельную опасность для самой демократии и стала прямой причиной разрушения демократии в большинстве стран. Поставив правильный диагноз, мы должны предпринять необходимые радикальные меры и реформы, даже если, спасая само существование демократии, мы несём определённые неизбежные риски.

Но нет никаких оснований опасаться, что такие злоупотребления будут сколько-нибудь значимыми. Как только будет введено в действие надлежащее демократическое законодательство — национальное и международное, нет никаких оснований полагать, что независимые суды не смогут надлежащим образом и в демократическом смысле справляться с ситуациями.
 
Невозможно сказать, можно ли вообще когда-нибудь прекратить <<войны>>. Вероятно, этого никогда не будет. Но этот тип войн — войн между нациями, отделёнными друг от друга искусственными границами, — может быть отменён, как только определённые части суверенитета, осуществляемого сегодня этими нациями, будут переданы вышестоящему органу. Религиозные войны были прекращены только тогда, когда народы начали осуществлять свои суверенные права независимо от церквей.

Из всех этих существующих тенденций вытекает революционный принцип предоставления частичного суверенитета национальностям на \textit{негеографической} основе.
 
Территория как основа для национальных суверенных государств, над которыми правят центральные правительства, была возможна только для великих наций с компактной и единой национальностью. Быстрый крах системы 1919 года доказывает невозможность организации малых наций в независимые государства на территориальной основе, где национальности смешиваются. Поправка законодательства о меньшинствах обернулась полным банкротством.

Та часть мира, в которой новая форма объединения является категорической необходимостью и должна быть предпринята попытка, — это та часть Европы, которая находится между Германией и Россией, Балтикой и Средиземноморьем. В этой части Европы, откуда проистекает большинство причин войн и где около двадцати наций живут в такой взаимопроникающей форме, что между ними невозможно провести национальных границ, формирование государств, основанных на национальности, абсолютно невозможно.
 
Единственным решением, приемлемым для всех вовлечённых национальностей, по-видимому, является формирование независимых национальных правительств в Варшаве, Праге, Будапеште, Белграде и всех других столицах, правительств, которые должны обладать властью не над определённой ограниченной территорией, а над определённой национальностью, независимо от территории, на которой они проживают. Естественно, суверенными правами в финансовых, военных и иностранных делах должно было бы заниматься выбранное ими федеральное правительство. Но в любом культурном и национальном вопросе румын, серб и венгр, живущие в одной деревне, должны быть в состоянии быть верными национальным правительствам Румынии, Сербии и Венгрии.

Аналогию этому решению можно найти в религиозных войнах, когда страны с различными религиями вели войны друг против друга до тех пор, пока превосходящая власть не позволила католикам, протестантам, мусульманам, православным грекам, баптистам, евреям и всем остальным следовать правилам своих собственных церквей без необходимости убивать друг друга по этой причине.
 
Другой важной частью мира, где формирование национальных суверенитетов на негеографической основе могло бы стать решением многовековой борьбы, является Индия.

Правительства различных конфликтующих стран, которые борются за престиж, власть, национальную независимость, равновесие и многие другие иллюзии, похоже, забывают, что, помимо всего этого, люди также хотят есть. Итак, после невероятной эволюции промышленного производства сегодня больше миллионов людей страдают от голода и нищеты, чем когда-либо прежде.
 
Нацизм и фашизм — это всего лишь симптомы мирового кризиса и упадка определённой системы, которая больше не имеет ничего общего с нынешними условиями. Если мы хотим продолжать массовое производство и делить этот крошечный мир на сотню водонепроницаемых отсеков, если мы хотим продолжать быть <<независимыми>>, <<суверенными>>, <<нейтральными>> и кем бы то ни было ещё, у нас будут войны, войны и войны.

Если мы верим, что <<демократия>> — это система, которая \textit{была} у нас в прошлом, и если мы хотим продолжать верить в те же устаревшие концепции, которые были всего лишь выражением первой попытки демократического порядка, установленного на основе реалий восемнадцатого века, то такая демократия будет разрушена подобно гнилое дерево во время бури.
 
Демократия не является и никогда не сможет быть закрытой жёсткой системой. Это её смерть. Любая закрытая жёсткая система должна приводить к войнам, революциям и диктатурам. Демократия нуждается в постоянной настройке. Её институты требуют постоянного обновления. Таким образом, демократия не может быть определена какой-либо системой институтов, существующих или подлежащих созданию.

Демократия — это атмосфера, единственная атмосфера, в которой современный человек может жить, процветать и прогрессировать.
 
Политическая организация, которая требуется для решения проблем войны и мира, свободы и рабства, является не отдалённой целью, а насущной необходимостью. Мы не можем терять много времени. Фундамент должен быть заложен сейчас, во время этой войны.

На самом деле, установление демократического миропорядка — это не что иное, как самое начало реальной работы, которую нам предстоит проделать: решение социальных проблем, проблем производства, распределения и потребления, проблемы общего повышения уровня жизни человеческой расы.
 
Только если мы будем иметь в виду эти огромные социальные и экономические проблемы, мы сможем увидеть стоящую перед нами политическую задачу в её реальной перспективе; это может придать нам смелости взяться за неё без промедления и решить её сейчас.

Если мы не сможем принять решение о создании сейчас единственно возможной политической структуры нашего времени, у нас не будет шанса решить ни одну из тех проблем, которые на самом деле являются реальными повседневными проблемами каждого мужчины и женщины и которые должны быть главной задачей всех правительств нашего времени.
