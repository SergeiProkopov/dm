\chapter{Независимость}

Стремление к свободе как естественное побуждение индивидуума находит свою параллель в стремлении к независимости в жизни наций. Действительно, стремление к национальной независимости является столь же мощным естественным побуждением в истории, как и стремление к индивидуальной свободе. Точно так же, как Французская революция была величайшим символическим событием в освобождении личности и провозглашении прав человека, величайшим символическим событием в обретении национальной независимости от иностранного правления была Американская революция и Декларация независимости. Эти две великие революции имели один и тот же корень и основывались на одних и тех же принципах. Декларация прав человека и гражданина и Декларация независимости были основаны на естественном праве [\textit{<<Laws of Nature>>}], на определённых неотъемлемых правах, на равенстве, свободе, братстве, а также на жизни, свободолюбии и стремлении к счастью.
 
Стремление к национальной независимости достигло огромных успехов в девятнадцатом веке и достигло своей кульминации в мирных договорах 1919 года, когда — по крайней мере, в Европе и Америке — каждая нация, даже самая маленькая, стала независимой. Основным принципом порядка, созданного в 1919 году, было то, что каждая нация, независимо от её величины, имела право организовывать свою независимую национальную жизнь, как Соединенные Штаты Америки, Великобритания и Французская Республика организовали свою национальную жизнь в соответствии с принципами восемнадцатого века. Чехи, литовцы, поляки и румыны стали независимыми, и в 1920 году больше наций, чем в любой другой период истории, добились полной независимости.

Что стало с этой широко распространенной независимостью?

Двадцать лет спустя больше наций, чем когда-либо прежде, было покорено либо путём прямого военного завоевания, либо другими средствами политического господства. И этот современный вид порабощения был более безжалостным, чем любая другая форма зависимости до великих революций за свободу и независимость.

Почему было невозможно такому множеству наций сохранить свою независимость дольше, чем всего лишь несколько лет?

Ответ на этот вопрос очень прост. Полная независимость всех наций создала в международной жизни такую анархическую ситуацию, которая была бы неизбежна в любом государстве, в котором существовала бы полная свобода личности, без каких-либо ограничений, без каких-либо законов, без принуждения.

Уилсон, полковник Хауз, Масарик, лорд Сесил, Леон Буржуа и другие энциклопедические составители  договоров 1919 года считали, что каждая нация была бы счастлива быть независимой и что благодаря сотрудничеству этих независимых государств, основанному на взаимной доброй воле, мы могли бы поддерживать мир, организовывать международную жизнь и развивать человеческий прогресс. Они полностью игнорировали огромное количество мощных импульсов в человеческой природе, столь же <<естественных>>, как стремление к свободе и независимости, которые, если их не регулировать, не контролировать и не сдерживать, приносят в жизнь <<независимых>> наций больше страданий, чем худший вид рабства.

Они полностью игнорировали тот факт, что политическая независимость каждого национального государства несовместима с текущими экономическими условиями. Итак, за короткий период между 1920 и 1940 годами мы стали свидетелями того, что мир, ставший безопасным для демократии, был неспособен предотвратить реакцию на демократические идеи, на идеи самоопределения, независимости и равенства, более бурную и пылкую, чем любое другое изменение в истории за столь короткий срок.

Независимость наций была табу, и ни одно независимое правительство не было готово брать на себя международные обязательства, опасаясь, что любое такое обязательство может поставить под угрозу независимость их страны. Не существовало обязательного международного права, и не было организации, которая могла бы заставить народы подчиняться такому закону. Следовательно, правительства независимых государств не доверяли друг другу и вели интриги друг против друга, точно так же, как это делали абсолютные монархи, когда они правили покоренными народами.

В каждой независимой нации доминировал глубокий, естественный импульс, который мы называем страхом и который является корнем столь многих злых человеческих поступков. Из-за страха, взаимного недоверия и зависти, которые никогда не удавалось искоренить никакими призывами к разуму, между нациями возникали естественные конфликты точно так же, как они возникали, когда они не были независимыми. В каждом случае, когда возникал такой конфликт, независимые правительства встречались, чтобы изучить вовлечённые факторы и определить, кто несёт ответственность. Поскольку для таких случаев не существовало закона и общепринятого определения, они обменивались своими собственными субъективными мнениями, каждый, конечно же, пытался отстаивать интересы своей нации. Очевидно, они так и не пришли ни к какому практическому выводу; решение каждого из этих конфликтов откладывалось, в результате чего число и серьёзность этих конфликтов росли в геометрической прогрессии.

Одни и те же юридические лица, <<представители>> независимых государств, были в международной жизни законодателями, судьями и исполнителями. Идея заключалась в том, что собрание представителей таких независимых правительств должно выполнять все три функции демократической международной жизни. Эта примитивная концепция международной жизни, основанная на национальной независимости, вела от конфликта к конфликту, пока в случае итальянской агрессии в Эфиопии мировое общественное мнение не было взбудоражено до такой степени, что некоторого рода действия стали неизбежными.

Впервые была предпринята попытка применить санкции против великой державы, нарушающей независимость другого государства. Процедура была основана на таком импровизированном сотрудничестве наций. Результатом стал полный провал. Никто не объяснил причины этого провала яснее, чем бывший президент Франции Мильеран в своей речи во французском Сенате 26 июня 1936 года. Французский государственный деятель-националист, призывающий к признанию завоеваний Муссолини, сказал: <<\ldots\ На самом деле причина провала санкций совершенно иная. Чтобы понять это, достаточно вспомнить, что в мире нет ни одной нации, которая согласилась бы вести войну, современную войну, кроме случая, когда затронуты её жизненные интересы, когда под угрозой находится её национальная независимость и её границы>>.

Эта аргументация является естественным выражением концепции независимости, которую мы утвердили в прошлом и которой большинство наших государственных деятелей придерживаются до сих пор.

Лучше всего это объяснил сенатор Бора, когда он был председателем комитета по международным отношениям американского Сената.

Сенатор от Айдахо сказал: <<В этом мире есть вещи, которых следует желать больше, чем мира, и одна из них — это нестеснённая, беспрепятственная и свободная от ограничений политическая независимость этой республики — право и власть определять в любом кризисе, когда этот кризис наступает, не сдерживаемая никакими предыдущими обязательствами, будущее этой республики — это курс, которому лучше всего следовать людям этой страны. Если мир не может быть достигнут без нашего отказа от этой свободы действий, тогда я не за мир>>.

Следует надеяться, что катастрофа, в которой мы оказались сегодня в результате такой интерпретации независимости, заставит некоторых наших лидеров задуматься об этой проблеме в текущих условиях, применительно к текущим веяниям и реалиям мира, и посмотреть, является ли выражение независимости, как это заложено сегодня в демократических конституциях, действительно ли помогает нам добиться лучшей жизни, свободы и стремления к счастью, ради которых оно было провозглашено.

Сегодня политическая независимость не имеет смысла без экономической независимости. И экономическая независимость, очевидно, не может быть достигнута ни одной нацией в отдельности. Мы привыкли считать, что в мире есть по крайней мере две или три крупные нации, которые могут называть себя экономически независимыми. Но нынешняя война показала, что даже Соединенные Штаты Америки, даже Советская Россия не являются экономически независимыми.

С учётом этих выявляющих фактов, доказывающих, что ни одна нация, даже самая могущественная, не может достичь экономической независимости, мы должны осознать, что на современном этапе истории политическая независимость больше не является идеалом, который может помочь нам в достижении нашей цели. Как практический институт, он разрушителен.
