\chapter*{Диалог Конфуция (Учителя) и Цзы-лу}
\begin{flushright}
\normalsize(КОНФУЦИЙ: <<ЛУНЬ ЮЙ>>, ГЛАВА XIII, 3) 
\end{flushright} 

Цзы-лу спросил: 

— Вэйский правитель намеревается привлечь вас к управлению государством. Что вы
сделаете прежде всего? 

Учитель ответил:

— Необходимо начать с исправления имен.\footnote{%
Цзы-лу служил у вэйского правителя Чу Гун-чжэ, который намеревался привлечь на службу и Конфуция, приехавшего в Вэй из Чу. Поэтому Цзы-лу стал расспрашивать Конфуция о том, что он сделает прежде всего, если вэйский правитель возьмет его на службу. Исправление имен в соответствии с действительностью (чжэн мин) — один из основных элементов социально-политической теории Конфуция. Смысл учения об исправлении имен состоит в том, что социальная роль каждого члена общества должна быть не номинальной, а реальной. Это значит, что государь, чиновник, отец, сын должны не только так называться, но и обладать всеми качествами, правами и обязанностями, вытекающими из этих названий. Появление учения об исправлении имен было вызвано серьёзными социально-политическими изменениями в китайском обществе, в результате которых название социальных ролей не соответствовало качествам, реальному положению или поведению их исполнителей. Так, вэйский правитель Чу Гун-чжэ, к которому приехал Конфуций, не был законным правителем, поскольку управление Вэй завещалось Лин-гуном не ему, а Ину — сыну Лин-гуна.}

Цзы-лу спросил:

— Вы начинаете издалека. Зачем нужно исправлять имена?

Учитель сказал:

— Как ты необразован, Ю! Благородный муж проявляет осторожность по отношению к тому, чего не знает. Если имена неправильны, то слова не имеют под собой оснований. Если слова не имеют под собой оснований, то дела не могут осуществляться. Если дела не могут осуществляться, то ритуал и музыка не процветают. Если ритуал и музыка не процветают, наказания не применяются надлежащим образом. Если наказания не применяются надлежащим образом, народ не знает, как себя вести. Поэтому благородный муж, давая имена, должен произносить их правильно, а то, что произносит, правильно осуществлять. В словах благородного мужа не должно быть ничего неправильного.
