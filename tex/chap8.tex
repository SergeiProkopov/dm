\chapter{Война}

Центральной проблемой всех наших разногласий является, конечно, проблема войны. Эта проблема так же стара, как сама история человечества. На самом деле история человечества — это не что иное, как история войн.
 
За исключением убеждённых милитаристов и приверженцев современной формы язычества, представленной фашистско-нацистскими движениями, подавляющее большинство людей всех рас испытывают глубокое чувство, что война — это нечто злое, неправильное, своего рода катастрофа, и все они желают мира.

Это чувство, вероятно, так же старо, как история человечества. Но, несмотря на тот факт, что подавляющее большинство человечества на протяжении тысячелетий ненавидело войну и стремилось к миру, история показывает, что за последние столетия на этой планете было меньше лет без войн, чем лет, в течение которых некоторые части человеческой расы вели войны.
 
Похоже, что есть доля истины в знаменитом определении Клаузевица о том, что война — это продолжение политики другими средствами. Если мы отбросим старые времена и посмотрим на развитие последнего столетия, начиная с окончания наполеоновских войн, то становится всё яснее и яснее, что распространение образования, науки, коммуникаций заставляет всё больше и больше людей верить в то, что война — это нечто такое, что необходимо прекратить. Ни одно правительство ни в одной из цивилизованных стран не смогло заручиться поддержкой большинства своего народа с помощью военной программы. Все правительства обещали мир. Все они должны были пообещать бороться против войны, и все они смогли втянуть свои народы в войну, только заставив их поверить, что на них напали и они просто защищались. Несмотря на это растущее стремление к международному согласию, человечество было втянуто в более разрушительные войны, чем когда-либо прежде.
 
Почему мы не можем остановить войны, если люди действительно хотят их прекратить?
 
Чтобы ответить на этот вопрос, мы должны сначала задать другой, более простой вопрос: что такое война?
 
Общепринятый ответ на этот вопрос заключается в том, что война — это борьба между двумя группами людей с оружием в руках.
 
Если мы так понимаем войну; если мы подразумеваем под войной просто убийство людей, уничтожение собственности друг друга, борьбу за какую-то цель, тогда мы никогда не сможем прекратить войны. Если мы интерпретируем войну таким примитивным образом, то наше желание прекратить её — детская утопия. Согласно этому определению, война — это применение силы нациями, и вы не можете отменить применение силы, которое исходит из глубоких инстинктов и страстей и заложено в самой человеческой природе.
 
Мы никогда не были в состоянии искоренить индивидуальное преступление, применение грубой силы между индивидуумами, хотя мы пытались сделать это с самих истоков организованного общества, с зарождения религии.
 
Но чего мы смогли достичь в отношении индивидуальных преступлений в организованном обществе, так это прояснить с помощью определённого законодательства, какие действия считаются преступлениями, и создать необходимую организацию, законодательство, юрисдикцию и полицейские органы, чтобы свести такие преступные действия к минимуму, а через наказание за преступления создать чувство индивидуальной безопасности граждан.
 
Война как борьба между народами, как взрыв человеческих страстей, как динамичный фактор человеческой истории не может быть и никогда не будет прекращена.
 
Но если мы примем во внимание эволюцию индивидуальных убийств, воровства, боевых действий внутри организованного государства, мы не сможем согласиться на такое упрощенное определение войн, которое сегодня общепринято во всем мире.
 
Мы проводим очень чёткое различие между человеком, который убивает кого-то, чтобы вытащить тысячу долларов из кармана своей жертвы, и человеком, который казнит кого-то на основании юридического документа, который мы называем судебным решением. Хотя эти два деяния с биологической точки зрения абсолютно идентичны, мы обычно не называем их обоих убийством.
 
Такое же различие должно быть принято и чётко сформулировано в отношении процессов убийств группами людей, которые мы обычно называем войнами.
 
Мы не сможем прекратить войну с помощью какой-либо вообразимой организации, точно так же, как мы не смогли прекратить убийство, несмотря на всю мощь организованной полиции. Возможно, мы были бы в состоянии свести международные войны к минимуму с помощью соответствующей организации народов, точно так же, как мы смогли свести случаи убийств в цивилизованном государстве к абсолютному минимуму.
 
Основой такой организации должно быть чёткое и безошибочное определение того, что мы понимаем под тем видом войны, который мы хотим прекратить. Мы должны проводить различие между \textit{законными} и \textit{незаконными} войнами. Нашей единственной возможностью покончить с незаконными войнами, по-видимому, является принятие и легализация определённых видов военных действий, к которым нам придётся прибегнуть, если мы хотим избавиться от таких разрушительных мировых войн, свидетелями которых наше поколение становилось дважды.
 
Прежде всего, мы должны отказаться от всех тех примитивных идей, которым мы следовали в течение последних десятилетий, чтобы <<очеловечить>> войны. Это пустая трата времени и крайне наивная концепция нашего нынешнего века. До тех пор, пока войны решались и велись монархами, в основном профессиональными армиями, было возможно установить определённые <<правила>> для таких войн, как если бы это были поединки в школе фехтования. Но поскольку в современных войнах к всеобщей воинской повинности привлекаются целые народы, все подобные правила невыполнимы и не имеют никакой ценности. Очевидно, что в любой такой войне каждая нация применит все виды оружия, которые, по её мнению, могут принести победу в кратчайшие сроки. Таким образом, все правила, установленные различными конвенциями в отношении применения определённых видов оружия, бомбардировок гражданского населения, ведения подводной войны, являются ничем иным, как принятием желаемого за действительное в мирное время, на что ни одна армия не обращает никакого внимания, участвуя в современной войне.
 
Мы могли бы предотвратить определённый тип международной войны с помощью определённой политики, законодательства и применения силы, но мы, конечно, никогда не сможем <<очеловечить>> войну, когда она разразится.
 
Такой же наивной идеей предотвращения войн является идея разоружения, которую так страстно отстаивали пацифисты с 1919 по 1935 год, вплоть до полного краха Конференции по разоружению. Полагать, что мы сможем поддерживать мир, уменьшив калибр орудий, тоннаж военных кораблей или количество обученных солдат, действительно наивно. Как будто у нас не было войн до появления 40 000-тонных дредноутов, до появления 16-дюймовых пушек и до появления многомиллионных армий.
 
Если мы собираемся поддерживать международное общество, состоящее из суверенных государств без какой-либо правовой организации, тогда у нас будут периодические войны, как это было в прошлом, независимо от того, какое оружие мы разрешаем им использовать, и мы можем быть уверены, что каждая нация будет использовать всё современное оружие, которое современная наука и промышленность могут предоставить им.
 
Мы никогда не сможем избавиться от войн путём разоружения. Разоружение может быть результатом только деятельности международной организации по предотвращению незаконных войн. На самом деле, если мы рассматриваем вооружения как причину войн, то урок, который преподаёт история, заключается в том, что только неравенство в вооружениях могло поддерживать мир в течение определённого времени. Равенство вооружений всегда означало и, вероятно, всегда будет означать войну.
 
Один из самых ошибочных выводов, который мы можем сделать из истории, состоит в том, чтобы рассматривать мир и войну как две разные вещи, как два противоположных полюса, как два условия, которые исключают друг друга. На самом деле, они выглядят колебаниями одного и того же аспекта человеческого общества, точно так же, как холод и жара — это температура, только в разной степени. Чтобы создать температуру, наиболее подходящую для человеческого организма, иногда мы должны добавить тепла, иногда мы должны уменьшить его. При хорошо организованном международном порядке, к которому мы стремимся, нам придётся время от времени предпринимать воинственные действия, чтобы поддерживать и укреплять социальное равновесие и международное спокойствие.
 
Самым весомым аргументом догматических пацифистов, сторонников теории разоружения и невмешательства было то, что <<вы не можете предотвратить войну, ведя её>>.
 
Это самый опасный софизм. На самом деле, единственный способ предотвратить незаконные и анархические войны — это вести определённого рода \textit{законную} войну, точно так же, как единственный способ бороться с преступностью и снижать её уровень — это совершать те же <<преступления>> на законных основаниях против преступников.
 
Эта легализация определённого типа войн не имеет ничего общего с понятием \textit{справедливой войны} [\textit{Bellum Justum}], веками использовавшимся в дебатах по международному праву.
 
Государственные деятели и юристы привыкли называть \textit{справедливую войну} войной возмездия, <<оправданной>> войной против государства, ответственного за незаконное деяние.
 
Этот термин является чисто субъективным понятием и не имеет практического смысла. На самом деле, все войны в истории были настолько подготовлены, что солдаты и нации, которые в них участвовали, были убеждены, что они сражаются на \textit{справедливой войне}. Любая война любой нации велась за <<правое дело>>, за <<оправданные национальные интересы>> и <<в целях самообороны>>. Действительно, эта теория оправдывает \textit{все войны} и, следовательно, является всего лишь софистическим аргументом, совершенно неудовлетворительным объяснением прошлых войн.
 
Законные войны, которые мы должны развязать, если хотим прекратить незаконные войны, предполагают существование международного законного порядка.
 
Они подразумевают силовые военные действия, предпринимаемые от имени сообщества, с ведома сообщества, для поддержания и охраны установленного законного порядка.
