\chapter{Люди или принципы устройства}

\sloppy Спустя двадцать лет после полного и решительного разгрома автократических и милитаристских принципов устройства и государств мир подвергается теперь величайшему в истории натиску тех же автократических, милитаристских сил. Спустя двадцать лет после величайшей победы демократических принципов устройства и демократических наций демократия по всей линии находится в обороне. Все демократические нации континентальной Европы покорены и завоёваны. Две самые могущественные единицы мира — Британское Содружество Наций и Соединённые Штаты Америки — вынуждены мобилизовать все свои ресурсы, чтобы защитить себя и избежать поражения и завоевания теми самыми антидемократическими силами, которые были повержены ими двадцать лет назад.

Что произошло за эти двадцать лет?

После победы в 1918 году демократические, свободолюбивые нации получили полный контроль и абсолютную власть на этой планете. Те силы, которые угрожали безопасности и спокойствию демократий, были не сильнее тех немногих преступников, которые всегда угрожают личной безопасности и общественному порядку внутри организованного государства.

Все стремились к миру. Все требовали разоружения. Все хотели расширенной свободы международной торговли. Все хотели лучшей организации международной экономической жизни, чтобы обеспечить каждой нации и каждому отдельному человеку больше благосостояния и большую безопасность, чем когда-либо прежде. Технические достижения, казалось, сделали все эти надежды реальностью. Несмотря на тот факт, что силы, поддерживавшие эти устремления, были десять к одному — может быть, пятьдесят к одному — против тех, кто выступал против них, начиная с 1930 года, год за годом, месяц за месяцем, мы неуклонно шли к увеличению вооружений, обострению антагонизма, большей бедности, сокращению торговли, к нетерпимости, гонениям, диктатурам, империализму — ко Второй мировой войне.

Молодой человек, все образование которого было основано на надеждах 1920-х годов и который внимательно следил шаг за шагом за событиями, приведшими ко Второй мировой войне, с трудом может понять это сбивающее с толку развитие событий, если только он не был полностью введен в заблуждение разлагающей пропагандой движения демагогических масс. Как бы он ни размышлял, ему, похоже, ничего не остаётся, кроме как признать, что он ничего не понимает в этом мире, и вспомнить слова, которые он выучил в школе — слова, произнесённые шведским канцлером Оксенштерна, когда он посылал своего сына в 1648 году в качестве полномочного министра в Вестфалию: <<Ты увидишь, сын мой, с какой малой мудростью управляется этот мир>>. Он должен со страхом спросить себя: \textit{Где} хотя бы эта мельчайшая часть мудрости, с которой мир должен управляться?

Самый простой способ объяснить этот быстрый и почти катастрофический упадок демократий за короткий период между 1920 и 1940 годами — это обвинить некоторых государственных деятелей, допустивших величайшие ошибки. Нам нравится обвинять в неэффективности либеральные режимы Нитти и Факта за приход Муссолини, и в слабости Веймарскую республику — за приход Гитлера. Раньше мы обвиняли Лоджа, Бору, Рида и других сенаторов-изоляционистов в торпедировании Лиги Наций ещё до её создания. Мы осуждали Клемансо, Пуанкаре и других французских националистов за то, что они мешали укреплению демократии в Германии. Мы считаем сэра Остина Чемберлена и тори в Англии ответственными за провал Женевского протокола. Мы считаем, что Макдональд, Болдуин и Хендерсон несут ответственность за разоружение и слабость Британии. Мы считаем, что в провале <<восточного Локарно>> виноваты Пилсудский и Бек. Мы обвинили сэра Сэмюэля Хора и Пьера Лаваля в предательстве Эфиопии и саботаже Лиги. Мы думаем, что Невилл Чемберлен, Даладье и Жорж Бонне — люди, совершившие мюнхенское преступление. Мы привыкли обвинять Леона Блюма и Пьера Кота в том, что они несут ответственность за слабость французских вооружённых сил. И мы обвиняли Чарльза Линдберга, сенатора Уилера, Ная, Джонсона и других в американском изоляционизме и в том, что они сделали невозможным адекватную подготовку Соединённых Штатов к отражению нападения.

Все эти персональные ответственности обоснованны, но только до определённой степени. Они не могут объяснить железную логику событий, которые имели место в этот странный период между двумя мировыми войнами. Они не могут дать объяснения, потому что очевидно, что ни один из этих государственных деятелей не был в состоянии контролировать события или руководить своей страной. Их контролировали и возглавляли более крупные силы. Некоторые примеры доказывают это. Когда общественное мнение в Англии и Франции восстало против политики Хора и Лаваля в эфиопском конфликте, их сменили сильнейшие противники — Энтони Иден и Ивон Дельбос. Вскоре после эфиопского инцидента возник ещё один конфликт тех же сил — испанская <<гражданская война>>. И в этом конфликте, вызвавшем самые сильные международные последствия, Иден и Дельбос совершили точно такие же ошибки, за которые они так яростно осуждали своих предшественников, и решили эту новую проблему точно так же, как справились бы с ней Хор и Лаваль. Таких примеров можно привести бесчисленное множество.

Бесспорно, что человеческие качества лидеров демократии в период между двумя войнами несопоставимы с качествами их предшественников девятнадцатого века. Если бы какое-либо частное предприятие управлялось так, как в последние годы управлялись великие демократические нации, руководители этого предприятия, безусловно, были бы уволены.

Похоже, что таков закон в истории великих наций или великих семей, что через определённое время возникает поколение, которому недостаёт тех качеств, которые необходимы для эффективного лидерства и непрерывного прогресса. Обычно это время, когда такие семьи и такие народы переживают упадок. Это трагический симптом, который возникает периодически и может быть преодолён только при помощи привнесения свежей крови.

Какими бы слабыми ни были государственные деятели демократических наций, они вовсе не были преступниками и слабоумными, как думает общественное воображение, судя о них по всем тем ошибкам, которые отождествляются с их именами. Длинная череда поражений, постепенное скатывание от мирового господства к подчинению, конечно, не могут быть адекватно объяснены недостатком искусства управлять государством у наших лидеров, не важно в какой мере они несут ответственность.

Другая школа мысли заставит нас поверить в то, что это изменение направления в истории человечества было вызвано великими динамическими силами, представленными нацистскими, фашистскими и другими тоталитарными движениями во всем мире. Они говорят, что Гитлер и Муссолини — тираны, страдающие манией величия, милитаристы, безжалостные демагоги, и что они и их шайки несут ответственность за беды и несчастья мира.

Каким бы очевидным и простым ни казалось такое объяснение, оно не выдерживает пристального рассмотрения.

Демократии были чрезвычайно могущественны, когда эти движения были начаты несколько лет назад ограниченным числом людей. Эти движения можно было бы остановить и уничтожить в бесчисленных случаях, затратив минимум стараний, энергии и силы. Ни одна демократия не смогла или не захотела этого сделать, хотя эти антидемократические силы никогда не предпринимали никаких попыток скрыть свой характер, свои программы и свои цели.

Обвинять нацистов или фашистов в единоличной ответственности за эту войну — всё равно что обвинять бактерию туберкулёза в том, что она вызывает туберкулёз.

Конечно, это так. Это её единственная и неизменная задача.

В течение многих лет мы знали о том, что делает вирус фашизма в крови нации, точно так же, как мы знаем, что делает бактерия туберкулёза с того момента, как она начинает развиваться в лёгких человека. Но если человек, инфицированный туберкулёзом, отказывается бороться с ним, грубо и умышленно игнорирует советы своего врача и позволяет бактерии поразить весь свой организм, то нельзя сказать, что его смерть была вызвана бактерией. Его смерть была вызвана скорее отсутствием у него собственной воли и способности бороться с этой бактерией.

Много лет назад нацисты и фашисты объявили, что они хотят уничтожить демократический образ жизни, христианскую религию и все формы индивидуальной свободы. Они осудили спокойствие и миролюбие, а также провозгласили войну естественным образом жизни.

Они воспитывали свою молодёжь на строго милитаристской основе. Они хотели готовиться к захватническим войнам и вести их, считая историческим законом, что сильные нации должны завоёвывать более слабые и править ими.

Ввиду того факта, что эта программа открыто провозглашалась в течение многих лет, что она повторялась в речах и на бумаге и то, что она была гласно и стремительно приведена в действие, невозможно в разгар этой мировой борьбы заявить, что ответственность лежит только на Гитлере. Он был смешным, жалким, неэффективным ничтожеством, когда впервые провозгласил свою программу, но великие демократии позволяли ему становиться всё сильнее и сильнее, пока он не стал настолько могущественным, что бросил вызов всему человечеству.

Нет, Гитлера нельзя считать единственным ответственным за эту мировую катастрофу. Он следует своему призванию. Он является инкарнацией зла, которое Провидение время от времени порождает, чтобы напомнить человеку, что спокойствие, свобода, счастье, терпимость, братство, прогресс, права человека, более того, даже право на существование — это не естественные дары природы, а плоды многовековой напряжённой борьбы, превзойдённой только борьбой, необходимой для сохранения этого наследия.

Почти каждый день мы слушаем ораторов или читаем статьи и книги комментаторов и политических обозревателей, выражающих сожаление по поводу длинной череды неудач, допущенных демократическими государствами за последние годы. Эти критики говорят: <<Если бы только сэр Джон Саймон выступил против японцев в Маньчжурии\ldots\ Если бы только Великобритания и Франция применили санкции против Италии, чтобы остановить её агрессию в Эфиопии\ldots\ Если бы только правительство Народного фронта во Франции в достаточной степени поддержало республиканцев в Испании\ldots\ Если бы только та или иная нация, если бы только тот или иной государственный деятель сделал то или это\ldots>>.

Всё это, очевидно, верно. Несомненно, что если бы демократические державы были готовы пожертвовать 5 000 солдат в Маньчжурии, 10 000 солдат, чтобы спасти Эфиопию, 20 000 солдат, чтобы помешать немцам и итальянцам установить марионеточный режим в Испании, если бы они были готовы рискнуть 50 000 солдатами, чтобы предотвратить оккупацию Австрии, тогда два или три года спустя Британской империи и Соединённым Штатам не потребовалось бы мобилизовывать всю свою мужскую силу, от 18 до 65 лет, и тратить сто миллиардов долларов или больше в год на вооружение.

Но, несмотря на все эти <<если бы только\ldots>>, несмотря на множество условных аргументов, фактом остаётся то, что ни одна демократия никогда не была в состоянии или не желала предпринять шаг, который мог бы предотвратить катастрофу. Почему?

Во многих случаях демократические власти могли эффективно действовать в нужный момент, но ни одна из них этого не сделала. Почему?

В ответе на этот вопрос, в точной оценке последних двадцати лет остаётся единственная надежда на лучшее будущее. Очевидно, что истинная причина катастрофы и невозможности её остановить должна лежать глубже, чем просто в неудаче определённых людей или определённых правительств.

В своём гневе и беспомощной злости мы называем Гитлера и его соратников <<уголовниками>>, <<преступниками>>, <<гангстерами>>. Но имеем ли мы право употреблять такие громкие слова?

Что такое <<преступник>>?

Нет другого возможного определения преступника, кроме того, что это человек, который нарушает какой-то существующий закон, принятый сообществом.

Но где же были законы, которые по общему мнению нарушил Гитлер? Были ли мы когда-либо готовы принять международное законодательство, чтобы его условия были обязательными для исполнения, а его нарушение можно было бы назвать преступлением? Таких законов никогда не существовало, и великие демократии отказались принимать их, когда у них была возможность устроить новый мир. Они отказались принимать какой-либо международный порядок, отличный от старого, созданного соглашениями и договорами между суверенными нациями. Такие соглашения и договоры не могут рассматриваться как законы, если только одна из сторон не готова применить силу, когда другая сторона их нарушает.

Но была ли какая-либо из демократических стран готова наказать Германию, Италию или Японию за нарушение тех примитивных законов, которые установлены международными договорами? Всякий раз, когда такой договор нарушался в одностороннем порядке, законопослушные демократии всегда быстро мирились со \textit{свершившимся фактом} [\textit{fait accompli}]. И с чисто моральной точки зрения трудно сказать, что является большим преступлением — нарушать закон или мириться с подобным нарушением.

В любой сфере жизни — в частной жизни, в семейной жизни, в деловой жизни — единственным признанным правильным способом ведения дел является продуманное, хорошо спланированное действие, которое выходит за рамки просто возможных преимуществ, выгодоприобретений и комфорта в течение следующих пяти минут после совершения действия. Любое частное лицо, любой бизнесмен, глава любой семьи должен вести свои дела таким образом, чтобы никакое событие, которое может произойти завтра или послезавтра, не застало его врасплох, неподготовленным и полностью растерянным.

Только в общественной жизни требуется, чтобы государственные деятели и правительства не выходили за рамки рассмотрения сиюминутных проблем настоящего. От них требуется ограничивать свои мысли и действия самыми неотложными и неизбежными вопросами дня, независимо от того, выгодны или невыгодны нации подобные \textit{для этого случая} [\textit{ad hoc}] решения сиюминутных проблем, рассматриваемые не только с точки зрения дня действий, но и в перспективе нескольких лет, или даже всего на несколько месяцев.

Во всех других сферах жизни мы называем такое ведение дел беспечным и легкомысленным. В политике мы называем это <<реалистичным>>.

В любой сфере жизни метод ведения дел, который выходит за рамки событий, которые могут произойти в последующие годы, мы называем мудрым и дальновидным. В государственных делах мы называем это <<нереалистичным>> и <<утопическим>>.

Демократии были принуждены своими врагами и сбитыми с толку лидерами к принятию этого софизма, который пытается убедить нас в том, что реалистичная политика — это не политика, которая создаёт реальности, а политика, которая в любой данный момент склоняется перед реальностями, созданными другими.

Полный хаос, в который сегодня погрузилось всё человечество, является результатом полного разрушения всех моральных и духовных ценностей, выработанных историей. После многих тысячелетий религиозного, морального, социального и политического прогресса мы снова очутились в мире столь же странном, столь же ненадёжном, столь же неизведанном, в каком, должно быть, Адам очутился после своего изгнания из Эдемского сада.

Нет другого выхода из этой неразберихи, кроме как начать всё сначала. Мы должны тщательно изучить все те элементарные принципы, на которых строится наша социальная и политическая жизнь. Существует около двадцати таких элементарных, базовых представлений, значение которых настолько запутано, что никто больше не знает, в чем их точное значение, как они влияют, в какой форме их можно применять и как с ними нужно обращаться. Мы должны начать с того, с чего, по словам Конфуция, должна начинаться любая государственная деятельность: чтобы придать словам, которые мы используем, их точное и безошибочное значение.

Если мы проанализируем эти элементарные принципы демократического порядка, мы увидим, что толкование, данное им нашими существующими основными законами, законадательством, правилами и обычаями, совершенно неадекватно, что в большинстве случаев они полностью противоречат самой сути этих принципов, и что мы позволяем идеалам, которые всегда были самой мощной движущей силой человеческого прогресса, и развитию вырождаться в бессмысленные слова, лишённые какого-либо содержания.

Мы увидим, что <<демократический порядок>>, который мы хотим защищать, больше вообще не является демократическим порядком — это всего лишь \textit{первая попытка} к нему, предпринятая в конце восемнадцатого века и опирающаяся на условия и соображения восемнадцатого века.

Полное отсутствие какой-либо реакции на Атлантическую хартию, полное отсутствие энтузиазма, с которым она была воспринята в демократических странах, доказывают, что у народов есть неосознаваемое чувство нереальности этого прежнего порядка.

Исследование реального значения этих основных принципов социальной и международной жизни раскроет, что нет демократического порядка, который нужно защищать, но \textit{есть} демократический мировой порядок, который нужно \textit{создавать}.

Верная интерпретация этих принципов, понимание их взаимного влияния дают нам единственно правильный диагноз загнивания того порядка, который был установлен в 1919 году и который должен был стать базисом мира, <<безопасного для демократии>>. Такой диагноз причин нынешнего хаоса сам по себе на 90 процентов состоит из терапевтических мер, необходимых для организации демократического мира, основанного на экономических, социальных и политических реалиях, какими они представляются в середине двадцатого века.

